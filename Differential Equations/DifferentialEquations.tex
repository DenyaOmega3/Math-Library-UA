\documentclass[a4paper, 10pt]{article}
\usepackage[margin=1in]{geometry}
\usepackage{amsfonts, 
			amsmath, 
			amssymb,
			amsthm,
			pgfplots,
			tikz,
			graphicx,
			caption,
			float,
			physics
			}
%\usepackage[none]{hyphenat}
\usepackage{fancyhdr} %create a custom header and footer
\usepackage[utf8]{inputenc}
\usepackage[english, main=ukrainian]{babel}
\usepackage{pgfplots}
\usepgfplotslibrary{fillbetween}
\usepackage{tikz}
\usepackage{graphicx}
\usepackage{physics}

\fancyhead{}
\fancyfoot{}
\parindent 0ex
\def\qed{$\blacksquare$}

\def\rightproof{$\boxed{\Rightarrow}$ }
\def\leftproof{$\boxed{\Leftarrow}$ }

\newtheoremstyle{theoremdd}% name of the style to be used
  {\topsep}% measure of space to leave above the theorem. E.g.: 3pt
  {\topsep}% measure of space to leave below the theorem. E.g.: 3pt
  {\normalfont}% name of font to use in the body of the theorem
  {0pt}% measure of space to indent
  {\bfseries}% name of head font
  {}% punctuation between head and body
  { }% space after theorem head; " " = normal interword space
  {\thmname{#1}\thmnumber{ #2}\textnormal{\thmnote{ \textbf{#3}\\}}}

\theoremstyle{theoremdd}
\newtheorem{theorem}{Theorem}[subsection]
  
\theoremstyle{theoremdd}
\newtheorem{definition}[theorem]{Definition}

\theoremstyle{theoremdd}
\newtheorem{samedef}[theorem]{Definition}

\theoremstyle{theoremdd}
\newtheorem{example}[theorem]{Example}

\theoremstyle{theoremdd}
\newtheorem{proposition}[theorem]{Proposition}

\theoremstyle{theoremdd}
\newtheorem{remark}[theorem]{Remark}

\theoremstyle{theoremdd}
\newtheorem{lemma}[theorem]{Lemma}

\theoremstyle{theoremdd}
\newtheorem{corollary}[theorem]{Corollary}

\makeatletter
\renewenvironment{proof}[1][Proof.\\]{\par
\pushQED{\hfill \qed}%
\normalfont \topsep6\p@\@plus6\p@\relax
\trivlist
\item\relax
{\bfseries
#1\@addpunct{.}}\hspace\labelsep\ignorespaces
}{%
\popQED\endtrivlist\@endpefalse
}
\makeatother

\newcommand\thref[1]{\textbf{Th.~\ref{#1}}}
\newcommand\exref[1]{\textbf{Ex.~\ref{#1}}}
\newcommand\prpref[1]{\textbf{Prp.~\ref{#1}}}
\newcommand\rmref[1]{\textbf{Rm.~\ref{#1}}}
\newcommand\lmref[1]{\textbf{Lm.~\ref{#1}}}
\newcommand\crlref[1]{\textbf{Crl.~\ref{#1}}}


\begin{document}
\tableofcontents
\newpage
    	
	\section{Диференціальні рівняння першого порядку}
	\subsection{Основні означення}
	\begin{definition}
 Задана область $D \subset \mathbb{R}^2$ - відкрита та однозв'язна; функція $f: D \rightarrow \mathbb{R}$\\
	\textbf{Диференціальним рівнянням першого порядку} називається наступний вираз:
	\begin{align}
	y' = f(x,y)
	\end{align}
	\end{definition}

	\begin{definition}
 \textbf{Розв'язком рівняння} (1) називається функція $\\ y= \varphi(x)$, що визначений та диференційований на відкритому інтервалі $I \subset \mathbb{R}$, графік якої міститься в області $D$ та задовольняє рівнянню (1)
 	\end{definition}
 	
	\begin{definition}
 Графіком розв'язку $y = \varphi(x)$ називається \textbf{інтегральною кривою}.
	\end{definition}
	
	\begin{example}
 Задане диференціальне рівняння: $\displaystyle y' = \frac{y}{x}$\\
	Розв'язком буде функція $\varphi(x)=Cx$, причому:\\
	$I=(0, +\infty)$, якщо $D=\{(x,y) \in \mathbb{R}^2: x>0\}$\\
	$I=(-\infty, 0)$, якщо $D=\{(x,y) \in \mathbb{R}^2: x<0\}$
	\end{example}
	
	\textbf{Геометричний зміст}\\
	Якщо $y=\varphi(x)$ - розв'язок рівняння (1), то $\displaystyle k = \frac{d\varphi(x)}{dx}$ - кутовий коефіцієнт до графіку функції $y=\varphi(x)$ в т. $x$
	
	\begin{remark}
 Розв'язки рівняння (1) іноді зрузчно розглядати як $x=\psi(y)$. Тоді рівняння (1) записують так:
	\begin{align*}
	x'=\frac{1}{f(x,y)}
	\end{align*}
	\end{remark}
	
	\begin{remark}
 В загальному випадку розв'язок рівняння (1) може розглядатись як неявна функція $F(x,y)=0$. Тоді рівняння (1) записують так:
	\begin{align*}
	dy - f(x,y)dx=0
	\end{align*}
	\end{remark}
	
	\begin{definition}
 \textbf{Рівнянням Пфаффа} називають наступне рівняння:
	\begin{align}
	M(x,y)dx + N(x,y)dy=0
	\end{align}
	\end{definition}

	(TODO)
	\bigskip \\
	\begin{definition}
 Задана область $D \subset \mathbb{R}^2$ - відкрита та однозв'язна; точка $(x_0, y_0) \in D$\\
	\textbf{Задачею Коші з початковою умовою} $(x_0,y_0)$ називається система рівнянь:
	\begin{align*}
	\begin{cases}
	\displaystyle \frac{dy}{dx}=f(x,y)\\
	y(x_0)=y_0
	\end{cases}
	\end{align*}
	\textbf{Розв'якзом} називають такий розв'язок функції першого рівняння $y=\varphi(x)$, для якого $\varphi(x_0)=y_0$
	\end{definition}
	
	\begin{example}
	$
	\begin{cases}
	\displaystyle \frac{dy}{dx}=x\\
	y(0)=0
	\end{cases}
	$\\
	Для нього існує єдиний розв'язок $\displaystyle y = \frac{x^2}{2}$, $I=\mathbb{R}$
	\end{example}
	
	\subsection{Деякі типи рівнянь першого порядку}
	\subsubsection{Рівняння з відокремлювальними змінними}
	Дане рівняння має наступний вигляд:
	\begin{align*}
	M_1(x)M_2(y)\,dx+N_1(x)N_2(y)\,dy=0
	\end{align*}
	або
	\begin{align*}
	\frac{dy}{dx} = f_1(x) f_2(y)
	\end{align*}
	$M_1, N_1, f_1$ - неперервні на $I_1$, а $M_2, N_2, f_2$ - неперервні на $I_2$.\\
	Розглянемо випадок, коли $M_2(y) \not\equiv 0$, $N_1(x) \not\equiv 0$. Тоді рівняння перепишеться наступним чином:\\
	$\displaystyle \frac{M_1(x)}{N_1(x)}\,dx = -\frac{N_2(y)}{M_2(y)}\,dy$\\
	Далі проінтегруємо її:\\
	$\displaystyle \int \frac{M_1(x)}{N_1(x)}\,dx = - \int \frac{N_2(y)}{M_2(y)}\,dy$\\
	Якщо в обох частинах ми знайшли первісні $F_1(x), F_2(y)$, то розв'язок задається неявно таким чином:\\
	$F_1(x)=-F_2(y)+C$\\
	При якихось $x_* \in I_1, y_* \in I_2$ таких, що $M_2(y_*) = 0$, $N_1(x_*)=0$, ці точки будуть відкинуті. Тобто розв'язок задається на меншому інтервалі
	
	\begin{example}
 Розв'язати рівняння: $\displaystyle \frac{dy}{dx}=2x \cos^2 y$\\
	Спочатку перевіримо, коли $\cos^2 y=0$. Тоді\\ $\displaystyle y_* \equiv \frac{\pi}{2} + \pi k, k \in \mathbb{Z}$ - розв'язок\\
	Тепер поділимо на $\cos ^2 y$:\\
	$\displaystyle \frac{dy}{\cos^2 y} = 2x\,dx$\\
	Інтегруємо обидві частини та маємо:\\
	$\displaystyle \tg y = x^2 + C$ на інтервалі $I = \mathbb{R} \setminus \{y_*\}$ - розв'язок\\
	Можна привести в іншому вигляді:\\
	$y = \arctg (x^2+C) + \pi m, m \in \mathbb{Z}$
	\end{example}
	
	\subsubsection{Однорідне рівняння}
	\begin{definition}
 Функція $f(x,y)$ називається \textbf{однорідною}, якщо:
	\begin{align*}
	\forall t \neq 0: f(tx, ty)=f(x,y)
	\end{align*}
	\end{definition}
	
	\begin{example}
 $\displaystyle f(x,y) = \frac{x^2+y^2}{xy}$ - однорідна, оскільки:\\
	$\displaystyle f(tx, ty) = \frac{t^2x^2+t^2y^2}{t^2xy} = \frac{x^2+y^2}{xy} = f(x,y)$
	\end{example}
	
	\begin{proposition}
 Функція $f(x,y)$ - однорідна $\iff \displaystyle \exists F(z): \\f(x,y)=F\left(\frac{y}{x}\right)$.
	\end{proposition}

	\begin{proof}
	\leftproof Дано: $\displaystyle \exists F(z): f(x,y)=F\left(\frac{y}{x}\right)$\\
	Тоді $\displaystyle f(tx,ty) = F\left(\frac{ty}{tx} \right) = F\left(\frac{y}{x} \right) = f(x,y)$. Отже, $f(x,y)$ - однорідна\\
	\\
	\rightproof Дано: $f(x,y)$ - однорідна\\
	Тоді $\displaystyle f(x,y) = f\left(x\cdot1, x \cdot \frac{y}{x} \right) = f\left(t\cdot1, t \cdot \frac{y}{x} \right) \overset{f(tx,ty)=f(x,y)}{=} f\left(1, \frac{y}{x}\right)$\\
	Тому оберемо $F(z) = f(1,z)$, що й завершує доведення 
	\end{proof}
	
	А тепер як розв'язувати. Дано стандартне диф. рівняння:
	\begin{align*}
	y'=f(x,y)
	\end{align*}
	Цього разу $f(x,y)$ - однорідна\\
	Скористаємось наступною заміною:
	\begin{align*}
	y=xz, \textrm{де } z=z(x)
	\end{align*}
	Знайдемо її похідну:\\
	$y' = z + xz'$\\
	Тоді наше початкове рівняння матиме вигляд:\\
	$\displaystyle z+xz'=f(x, xz) \overset{\textrm{однорідна}}{=} f(1,z) \overset{\textrm{позначу}}{=}g(z)$\\
	$xz'=g(z)-z$\\
	Прийшли до рівняння з відокремлювальними змінними\\
	$\displaystyle \frac{dx}{x} = \frac{dz}{g(z)-z} ...$\\
	Після інтегрування замінюємо: $\displaystyle z = \frac{y}{x}$. Ну а далі як вийде
	
	\begin{example}
 Розв'язати рівняння: $\displaystyle y' = \frac{x+y}{x-y}$\\
	Можна помітити, що $\displaystyle \frac{tx+ty}{tx-ty} = \frac{x+y}{x-y}$. Тобто ця функція - однорідна. Тому робимо заміну:\\
	$y = xz, z=z(x) \Rightarrow y' = z + xz'$\\
	$\displaystyle z+xz'=\frac{1+z}{1-z} \Rightarrow xz'=\frac{z^2+1}{1-z} \Rightarrow \frac{dx}{x} = \frac{1-z}{z^2+1} \,dz$\\
	$\displaystyle \ln |x| + C = \arctg z - \frac{1}{2} \ln (z^2+1) |\cdot 2$\\
	$\displaystyle \ln x^2 + \ln (z^2+1) C' = 2 \arctg z$\\
	$\displaystyle z = \frac{y}{x}$\\
	$\displaystyle 2 \arctg \frac{y}{x} = \ln C\sqrt{x^2+y^2}$
	\end{example}

	
	\subsubsection{Лінійне рівняння}
	Розглядується рівняння наступного вигляду (TODO):
	\begin{align*}
	y' + a(x)y = b(x)
	\end{align*}
	де $a,b \in C(I)$.\\
	При $b(x) \equiv 0$ таке рівняння називають \textbf{однорідним}. В іншому випадку - \textbf{неоднорідним}\\
	Розв'язок даного рівняння буде складатись так:\\ $y = y_{\textrm{g.h.}} + y_{\textrm{p.inh.}}$\\
	$y_{\textrm{g.h.}}$ - загальний розв'язок однорідного диф. рівняння\\
	$y_{\textrm{p.inh.}}$ - частковий розв'язок неоднорідного диф. рівняння\\
	Знайдемо $y_{\textrm{g.h.}}$:\\
	$y'+a(x)y=0 \Rightarrow \displaystyle \frac{dy}{y} = -a(x)\,dx \Rightarrow \ln |y| = -A(x) + C$\\ де $\displaystyle A(x) = \int a(x)\,dx$\\
	$\displaystyle \Rightarrow y_{\textrm{g.h.}} = Ce^{-A(x)}$\\
	Наступним кроком буде знайти $y_{\textrm{p.inh.}}$. Це можна зробити, якщо в початкове рівняння підставити цей раз наступне:\\
	$\displaystyle y = C(x)e^{-A(x)}$, тут $C(x)$ - функція\\
	Підставляючи, отримаємо задачу - знайти функція $C(x)$. Там рівняння буде з відокремленою змінною. Після чого ми підставляємо $C(x)$ в\\ $y = C(x)e^{-A(x)} = y_{\textrm{p.inh}}$
	
	\begin{example}
 Розв'язати рівняння: $\displaystyle y' + 2xy = xe^{-x^2}$\\
	Тут буде демонструватись ті самі кроки\\
	Спочатку знайдемо загальний однорідний розв'язок:\\
	$y'+2xy=0 \Rightarrow \displaystyle \frac{dy}{y} = -2x \,dx \Rightarrow \ln |y| = -x^2 + C \Rightarrow y_{\textrm{g.h.}} = Ce^{-x^2}$\\
	Задамо новий розв'язок: $y = C(x)e^{-x^2}$ та підставимо в початкове:\\
	$C'(x)e^{-x^2} + C(x)e^{-x^2}(-2x) + 2xC(x)e^{-x^2} = xe^{-x^2}$\\
	$C'(x) = x \Rightarrow \displaystyle C(x) = \frac{x^2}{2} + C_0$\\
	Остаточно отримаємо, що:\\
	$\displaystyle y = e^{-x^2} \left(\frac{x^2}{2} + C_0 \right) = e^{-x^2}C_0 + e^{-x^2} \frac{x^2}{2}$\\
	\bigskip \\
	Інший варіант розв'язку такого рівняння - це метод Бернуллі\\
	Заміна: $y = uv$, де $u=u(x), v=v(x)$\\
	$y' = u'v+v'u$\\
	$\Rightarrow u'v + v'u + a(x)uv = b(x) \Rightarrow u'v + u(v'+a(x)v)=b(x)$\\
	Далі ми $v'+a(x)v = 0$. Там достатньо одного розв'язку. Отримавши $v$, можемо знайти звідси $u$
	\end{example}

	
	\subsubsection{Рівняння Бернуллі}
	Розглядується рівняння наступного вигляду:
	\begin{align*}
	y' + a(x)y = b(x)y^{\lambda}, \lambda \in \mathbb{R} \setminus \{0,1\}
	\end{align*}
	\begin{remark}
 При $\lambda = 0$ рівняння буде лінійним.\\
	При $\lambda = 1$ рівняння буде з відокремленими змінними
	\end{remark}
	
	Якщо $\lambda > 0$, то $y \equiv 0$ буде також розв'язком\\
	
	
	А далі ділимо обидві частини на $y^{\lambda}$:\\
	$\displaystyle y'y^{-\lambda} + a(x)y^{-\lambda+1} = b(x)$\\
	Робимо заміну: $z = y^{-\lambda+1}$, $z=z(x)$\\
	$z' = (-\lambda + 1)y^{-\lambda}y'$\\
	$\Rightarrow \displaystyle \frac{z'}{1-\lambda} + a(x)z=b(x)$\\
	А далі вже розв'язується лінійне рівняння...
	
	\begin{example}
 Розв'язати рівняння: $\displaystyle y' + \frac{y}{x+1} + y^2 = 0$\\
	(тут $\lambda = 2$)\\
	$y \equiv 0$ - розв'язок. Далі ділимо на $y^2$:\\
	$\displaystyle y'y^{-2} + \frac{y^{-1}}{x+1} + 1 = 0$\\
	Заміна: $z = y^{-1} \Rightarrow z' = -y^{-2}y'$\\
	$\displaystyle -z' + \frac{z}{x+1} + 1 = 0$\\
	Заміна 2: $z = uv \Rightarrow z' = u'v+v'u$\\
	$\displaystyle -u'v - v'u + \frac{uv}{x+1} + 1 = 0 \Rightarrow u'v - u\underset{=0}{(\frac{v}{x+1}-v')} = 1$\\
	$\displaystyle \frac{v}{x+1} - v' = 0 \Rightarrow \cdots \Rightarrow v = \frac{1}{x+1}$\\
	$\displaystyle u'\frac{1}{x+1} = 1 \Rightarrow \cdots \Rightarrow u = \frac{1}{2}(x+1)^2 + C$\\
	Зворотня заміна 2: $\displaystyle z = uv = \frac{x+1}{2} + \frac{C}{x+1}$\\
	Зворотня заміна 1: $\displaystyle z = y^{-1} = \frac{x+1}{2} + \frac{C}{x+1} = \frac{(x+1)^2+2C}{2(x+1)}$\\
	Остаточно отримаємо:\\
	$\displaystyle y = \frac{2(x+1)}{(x+1)^2 + 2C}$\\
	$y \equiv 0$
	\end{example}

	
	\subsubsection{Рівняння, що можна звести до однорідного}
	Нехай задане ось таке рівняння:
	\begin{align*}
	y' = \frac{a_1 x + b_1 y + c_1}{a_2 x + b_2 y + c_2}
	\end{align*}
	Оскільки ми "хочемо" \hspace{0.1cm} звести це рівняння до однорідної, то нам необхідно перетворити цю дріб на наступний вигляд
	\begin{align*}
	Y' = \frac{\alpha_1 X + \beta_1 Y}{\alpha_2 X + \beta_2 Y}
	\end{align*}
	Можна компоненти дробу розглянути як два рівняння прямої:\\
	$\begin{cases}
	\alpha_1 X + \beta_1 Y = 0\\
	\alpha_2 X + \beta_2 Y = 0
	\end{cases}
	$\\
	та помітити, що в них існує єдиний розв'язок $(0,0)$.\\ \\
	В нашому конкретному випадку:\\
	$\begin{cases}
	a_1 x + b_1 y + c_1 = 0\\
	a_2 x + b_2 y + c_2 = 0
	\end{cases}
	$\\
	Система матиме єдиний розв'язок при умові, що $\begin{vmatrix} a_1 & b_1 \\ a_2 & b_2 \end{vmatrix} \neq 0$. Скажімо $(x_0, y_0)$. Тепер для однорідного вигляду я хочу таку заміну на $x$ та $y$ закласти, щоб згодом вони перетнулись в т. $(0,0)$.\\
	Тоді заміна:\\
	$\begin{cases}
	x = X + x_0\\
	y = Y + y_0
	\end{cases} \Rightarrow
	\begin{cases}
	dx = dX\\
	dy = dY
	\end{cases}
	\Rightarrow \displaystyle \frac{dY}{dX} = Y'
	$\\
	Тому ми отримали рівнання, яку ми "захотіли". І якщо погратись з алгеброю, то буде саме такий вираз:\\
	$\displaystyle Y' = \frac{a_1 X + b_1 Y}{a_2 X + b_2 Y}$\\
	А далі це однорідне рівняння...
	
	\begin{example}
 Розв'язати рівняння $\displaystyle y' = \frac{2y-x-5}{2x-y+4}$\\
	Одразу зауважу, що $\begin{vmatrix} -1 & 2 \\ 2 & -1 \end{vmatrix} \neq 0$\\
	Знайдемо точку перетину цих прямих:\\
	$\begin{cases}
	- x + 2y - 5 = 0\\
	2 x - y + 4= 0
	\end{cases} \Rightarrow
	\begin{cases}
	x = -1\\
	y = 2
	\end{cases} 
	$\\
	Заміна:
	$\begin{cases}
	X = x + 1\\
	Y = y -2
	\end{cases} \Rightarrow
	\begin{cases}
	dX = dx\\
	dY = dy
	\end{cases}
	$\\
	$\displaystyle Y' = \frac{2Y-X}{2X-Y}$\\
	Заміна 2: $Y=ZX, Z=Z(X) \Rightarrow Y' = Z + XZ'$\\
	$\displaystyle Z + XZ' = \frac{2Z-1}{2-Z} \Rightarrow XZ' = \frac{Z^2-1}{2-Z} \Rightarrow \frac{dX}{X} = \frac{2-Z}{Z^2-1}\,dZ$\\
	$\displaystyle \ln|X| + C= \frac{1}{2} \ln |Z-1| - \frac{3}{2} \ln |Z+1|$\\
	$\displaystyle \ln(CX)^2 = \ln \abs{\frac{Z-1}{(Z+1)^3}}$\\
	$\displaystyle CX^2 = \frac{Z-1}{(Z+1)^3}$\\
	Зворотня заміна 2:\\
	$\displaystyle CX^2 = \frac{\frac{Y}{X}-1}{(\frac{Y}{X}+1)^3} = \frac{YX^2-X^3}{(Y+X)^3}$\\
	$\displaystyle C = \frac{Y-X}{(Y+X)^3}$\\
	Зворотня заміна і остаточна відповідь:\\
	$\displaystyle C = \frac{y-x-3}{(y+x-1)^3}$
	\end{example}
	
	\subsection{Задача Коші}
	\begin{definition}
 Задана область $D \subset \mathbb{R}^2$ та функція $f: D \rightarrow \mathbb{R}$\\
	Така функція \textbf{задовольняє умові Ліпшиця відносно} $y$, якщо
	\begin{align*}
	\exists L>0: \abs{f(x,y_1) - f(x,y_2)} \leq L\abs{y_1-y_2}
	\end{align*}
	\end{definition}
	
	\begin{proposition}
 Задан прямокутник $\Pi = I_a \cross I_b \subset \mathbb{R}^2$. Відомо, що функція $f(x,y)$ має часткову похідну $f'_y$, яка обмежена в $D$. Тоді $f$ задовільняє умові Ліпшица в $D$, причому $\displaystyle L = \sup_{(x,y) \in D} |f'_y(x,y)|$
	\end{proposition}

	\begin{proof}
	Зафіксуємо довільне значення $x$\\
	За теоремою Лагранжа, $\exists \xi \in (y_1, y_2):\\f(x,y_1) - f(x,y_2) = f'_y(x, \xi)\cdot(y_1-y_2)$\\
	Тоді $\displaystyle \abs{f(x,y_1) - f(x,y_2)} = \abs{f'_y{x, \xi}} \abs{y_1 - y_2} \leq \underset{\displaystyle \underset{L}{\rotatebox[origin=c]{90}{=}}}{\sup_{(x,y) \in D} |f'_y(x,y)|} \abs{y_1-y_2}$ 
\end{proof}
	
	А тепер розглянемо задачу Коші:
	\begin{align*}
	\begin{cases}
	\displaystyle \frac{dy}{dx}=f(x,y)\\
	y(x_0)=y_0
	\end{cases}
	\end{align*}
	де $f: \Pi = I_a \cross I_b \rightarrow \mathbb{R}$\\
	$I_a = [x_0-a, x_0+a], I_b = [y_0-b, y_0+b]$\\
	Наша головна мета: дізнатись, чи буде розв'язок задачі Коші єдиним взагалі і за якими умовами
	
	\begin{lemma}
 Функція $y(x)$ - розв'язок задачі Коші $\iff$ $y(x)$ задовільняє інтегральному рівнянню:
	\begin{align*}
	y(x) = y_0 + \int_{x_0}^x f(t,y(t))\,dt
	\end{align*}
	Причому $y: I \subset I_a \rightarrow I_b$ - диференційована
	\end{lemma}
	
	\begin{proof}
	\rightproof Дано: $y(x)$ - розв'язок задачі Коші, тобто\\
	$\displaystyle y'(t) = f(t,y(t)) \Rightarrow \int_{x_0}^x y'(t)\,dt = \int_{x_0}^x f(t,y(t))\,dt \Rightarrow y(x) = y_0 +  \int_{x_0}^x f(t,y(t))\,dt$
	\\
	\\
	\leftproof Дано: $y(x)$ - розв'язок інтегрального рівняння\\
	Продиференціюємо з обох сторін:\\
	$\displaystyle y'(x) = 0 + f(x,y(x)) = f(x,y(x))$\\
	Більш того, $\displaystyle y(x_0) = y_0 + \int_{x_0}^{x_0} f(t,y(t))\,dt = y_0$\\
	Отримали, що $y(x)$ - розв'язок нашої задачі Коші 
	\end{proof}
	Ця лема знадобиться, оскільки розв'язок задачі Коші $y_*(x)$ ми будемо знаходити як границю рівномірно збіжної послідовності функцій \\$\{y_n(x), n \geq 1\}$ так, що:\\
	$\displaystyle y_n(x) = y_0 + \int_{x_0}^x f(t,y_{n-1}(t))\,dt$  $\forall n \in \mathbb{N}$\\
	$y_0(x) \equiv y_0$
	
	\begin{theorem}[Теорема Пікара]
	Задана функція $f: \Pi \rightarrow \mathbb{R}$ така, що $f \in C(\Pi)$ та під умовою Ліпшиця відносно $y$\\
	Тоді задача Коші містить \underline{єдиний} розв'язок $y_*(x)$ на інтервалі \\ $I_h = [x_0 - h, x_0 + h]$
	\end{theorem}
	
	\begin{proof}
	\textbf{Частина 1. Існування}\\
	1) Доведемо \textbf{(MI)}, що всі $y_n$ знаходяться в прямокутнику $I_h \cross I_b \subset \Pi$, тобто $\forall n \geq1: \forall x \in I_h: |y_n(x)-y_0| \leq b$\\
	$n = 1 \Rightarrow \displaystyle |y_1(x) - y_0(x)| = \abs{y_0 + \int_{x_0}^x f(t,y_0)\,dt - y_0} \leq \abs{\int_{x_0}^x |f(t,y_0)|\,dt} \leq$\\
	Оскільки $f \in C(\Pi)$, то вона взагалі є обмеженою, тому $\exists M = \displaystyle \max_{(x,y)\in \Pi} |f(x,y)|$\\
	$\displaystyle \leq M |x-x_0| \leq Mh \leq M \frac{b}{M}=b$\\
	Нехай умова виконується для фіксованого $n$. Перевіримо твердження для $n+1$:\\
	$\displaystyle |y_{n+1}(x)-y_0| = \abs{\int_{x_0}^x f(t,y_n(t))\,dt} \leq \abs{\int_{x_0}^x |f(t,y_{n}(t))|\,dt} \leq$\\
	За припущенням, $y_n$ вже лежить в заданому прямокутнику. Тому $f(x,y_n(x))$ є також обмеженою\\
	$\displaystyle \leq M|x-x_0| \leq Mh \leq M\frac{b}{M}= b$\\
	Отже, всі $y_n$ лежать в прямокутнику $I_h \cross I_b$\\
	\\
	2) Доведемо, що послідовність $\{y_n(x), n \geq 1\}$ рівномірно збігається\\
	Зауважимо, що\\
	$y_n(x) = y_0 + (y_1(x) - y_0) + (y_2(x)-y_1(x)) + \cdots + (y_n(x)-y_{n-1}(x))$\\
	Розглянемо ряд $\displaystyle \sum_{k=1}^{\infty} \left(y_k(x) -y_{k-1}(x)\right)$. \\
	Спробуємо довести збіжність критерієм Вейерштрасса, тобто ми оцінимо $|y_k(x) - y_{k-1}(x)|$ $\forall x \in I_h$ таким чином, щоб було число\\
	$\displaystyle |y_1(x) - y_0(x)| \overset{\textrm{п. 1)}}{\leq} M|x-x_0| = M \frac{|x-x_0|}{1!}$\\
	$\displaystyle |y_2(x) - y_1(x)| = \abs{\int_{x_0}^x f(t,y_1(t)) - f(t,y_0(t))\,dt} \leq \abs{\int_{x_0}^x \abs{f(t,y_1(t)) - f(t,y_0(t))}\,dt} \leq \\ \overset{\textrm{умова Ліпшиця}}{\leq} \abs{\int_{x_0}^x L|y_1(t)-y_0|\,dt} \leq \abs{\int_{x_0}^x LM|t-x_0|\,dt} = ML\frac{|x-x_0|^2}{2!}$\\
	\\
	...\\
	\\
	Використовуючи \textbf{MI}, отримаємо таку оцінку $\forall k \geq 1$ і $\forall x \in I_h$:\\
	$\displaystyle |y_k(x) - y_{k-1}(x)| \leq ML^{k-1} \frac{|x-x_0|^k}{k!} \leq ML^{k-1} \frac{h^k}{k!}$\\
	Отримаємо мажорантний ряд $\displaystyle \sum_{k=1}^{\infty} ML^{k-1} \frac{h^k}{k!}$. Перевіримо на збіжність:\\
	$\displaystyle \sum_{k=1}^{\infty} ML^{k-1} \frac{h^k}{k!} = \sum_{k=1}^{\infty} \frac{M}{L} \frac{(Lh)^k}{k!} = \frac{M}{L} \left(e^{Lh}-1\right)$ - збіжний\\
	Отже, підсумовуючи, отримаємо, що $\displaystyle y_n(x)^\rightarrow_\rightarrow y_*(x)$
	\bigskip \\
	3) Доведемо, що $\displaystyle y_*(x) = \lim_{n \to \infty} y_n(x)$ також знаходиться в прямокутнику $I_h \cross I_b$\\
	Згідно з п.1), можемо отримати, що:\\
	$\displaystyle |y_*(x) - y_0| = |\lim_{n \to \infty} y_n(x) - y_0| = |\lim_{n \to \infty} (y_n(x) -y_0)| = \lim_{n \to \infty} |y_n(x)-y_0| \leq \\ \leq \lim_{n \to \infty} b = b$\\
	\bigskip \\
	4) Доведемо, що $y_*(x) \in C(\Pi)$ та є розв'язком задачі Коші\\
	Оскільки $y_0(x), f(x,y) \in C(\Pi)$, то $f(x,y_0(x)) \in C(\Pi)$\\
	Тоді $y_1(x) \in C(\Pi)$, а за \textbf{МІ}, $y_n(x) \in C(\Pi)$. І нарешті, через рівномірну збіжність, $y_*(x) \in C(\Pi)$\\
	\\
	$\displaystyle y_*(x) = \lim_{n \to \infty} y_n(x) =\lim_{n \to \infty} \left(y_0 + \int_{x_0}^x f(t,y_{n-1}(t))\,dt \right) = y_0 + \int_{x_0}^x f(t, \lim_{n \to \infty} y_{n-1}(t))\,dt = \\ = y_0 + \int_{x_0}^x f(t, y_*(t))\,dt$\\
	Тоді за лемою, $y_*(x)$ - розв'язок задачі Коші. \textbf{Кінець частини 1.}\\
	\bigskip \\
	\textbf{Частина 2. Єдиність}\\
	!Вважаємо, що існують два розв'язки задачі Коші: $y_*(x), y_{**}(x)$.\\
	Розглянемо функцію $z(x) = y_{**}(x) - y_{*}(x)$ та оцінимо її:\\
	$\displaystyle |z(x)| = \abs{\int_{x_0}^x f(t,y_{**}(t))-f(t,y_{*}(t))\,dt} \leq \abs{\int_{x_0}^x \abs{f(t,y_{**}(t))-f(t,y_{*}(t))} \,dt} \leq \abs{\int_{x_0}^x L\abs{y_{**}(t) - y_{*}(t)} \,dt} = \abs{\int_{x_0}^x L\abs{z(t)} \,dt} \leq LM'|x-x_0| \leq LM'h$
	(TODO) 
\end{proof}
	\newpage
	
	
	\section{Диференціальні рівняння n-го порядку}
	\subsection{Основні означення}
	\begin{definition}
 Задана область $D \subset \mathbb{R}^{n+1}$ - відкрита та однозв'язна; функція $f: D \rightarrow \mathbb{R}$\\
	\textbf{Диференціальним рівнянням n-го порядку} називається наступний вираз:
	\begin{align}
	y^{(n)} = f(x,y,y',\cdots,y^{(n-1)})
	\end{align}
	\end{definition}

	\begin{definition}
 \textbf{Розв'язком рівняння} (3) називається функція \\ $y= \varphi(x)$, що визначений та диференційований $n$ разів на відкритому інтервалі $I \subset \mathbb{R}$, всі похідні якого містяться в області $D$ та задовольняє рівнянню (3)
 	\end{definition}
 	
	\begin{example}
 Задане диференціальне рівняння: $y'' = e^x$\\
	Розв'язком буде функція $\varphi(x) = e^x + C_0x + C_1$ на інтервалі $I = \mathbb{R}$
	\end{example}
	
	\begin{definition}
 Задана область $D \subset \mathbb{R}^{n+1}$ - відкрита та однозв'язна; точка $(x_0, y_0, y_0', \cdots, y_0^{(n-1)}) \in D$\\
	\textbf{Задачею Коші з початковою умовою} $(x_0, y_0, y_0', \cdots, y_0^{(n-1)})$ називається система рівнянь:
	\begin{align*}
	\begin{cases}
	\displaystyle y^{(n)} = f(x,y,y',\cdots,y^{(n-1)})\\
	y(x_0)=y_0\\
	y'(x_0) = y_0'\\
	\cdots\\
	y^{(n-1)}(x_0) = y_0^{(n-1)}
	\end{cases}
	\end{align*}
	\textbf{Розв'якзом} називають такий розв'язок функції першого рівняння $\\y=\varphi(x)$, для якого $\varphi(x_0)=y_0$, $\varphi'(x_0)=y_0'$, $\cdots$, $\varphi^{(n-1)}(x_0)=y_0^{(n-1)}$
	\end{definition}
	
	\begin{example}
 	$\begin{cases}
	y'' = e^x\\
	y(0) = 1\\
	y'(0) = 1
	\end{cases}$\\
	Вже отримали, що $y = e^x + C_0 x + C_1$. Тоді якщо знайти всі похідні та підставити значення, то отримаємо:\\
	$\begin{cases}
	e^0 + C_1 = 1\\
	e^0 + C_0 = 1\\
	\end{cases} \Rightarrow C_0 = 0, C_1 = 0
	$. Отже, $y=e^x$ - розв'язок задачі Коші
	\end{example}
	
	\subsection{Задача Коші}
	\begin{definition}
 Задана область $D \subset \mathbb{R}^{n+1}$ та функція $f: D \rightarrow 	\mathbb{R}$\\
	Така функція \textbf{задовольняє умові Ліпшиця відносно} $y,y',\cdots,y^{(n-1)}$, якщо
	\begin{align*}
	\exists L>0: \abs{f(x,y_1, y'_1,\dots, y^{(n-1)}_1) - f(x,y_2, y'_2,\dots, y^{(n-1)}_2)} \leq \\ \leq L\left(\abs{y_1-y_2}+\abs{y'_1-y'_2}+\dots+\abs{y^{n-1}_1-y^{(n-1)}_2}\right)
	\end{align*}	
	\end{definition}
	\begin{proposition}
 Задан прямокутник $\Pi = I_a \cross \Pi_b \subset \mathbb{R}^{n+1}$. Відомо, що функція $f(x,y,y',\cdots,y^{(n-1)})$ має всі частні неперервні похідні в $\Pi$. Тоді $f$ задовільняє умові Ліпшиця, причому $L=\max\{L_1,L_2, \dots, L_{n-1}\}$, $\displaystyle L_i = \sup_{(x,y,y',\cdots,y^{(n-1)})} \abs{f'_{y^{(i-1)}} (x,y,y',\dots,y^{(n-1)})}$
	\end{proposition}

	\begin{proof}
	Ми розглянемо функціяю $g: [0,1] \rightarrow \mathbb{R}$ таку, що:\\
	$\displaystyle g(t) = f(x, (1-t)y_1 + ty_2, (1-t)y'_1+ty'_2, \dots, (1-t)y^{(n-1)}_1+ty^{(n-1)}_2)$\\
	Зокрема отримаємо:\\
	$g(0) = f(x,y_1, y'_1, \dots, y^{(n-1)}_1)$\\
	$g(1) = f(x,y_2, y'_2, \dots, y^{(n-1)}_2)$\\
	Часткові похідні неперервні, тому $g \in C^1([0,1])$. За теоремою Лагранжа,\\
	$\exists \xi \in (0,1): g(0) - g(1) = -g'(\xi)$\\
	де $\displaystyle g'(t) = \frac{\partial f}{\partial x} \frac{dx}{dt} +  \frac{\partial f}{\partial u} \frac{du}{dt} + \dots +  \frac{\partial f}{\partial u^{(n-1)}} \frac{du^{(n-1)}}{dt} \boxed{=}$\\
	$u = (1-t)y_1+ty_2$, $\dots$, $u^{(n-1)} = (1-t)y^{(n-1)}_1 + ty^{(n-1)}_2$\\
	$\displaystyle \boxed{=} \frac{\partial f}{\partial u} (y_2-y_1) + \dots +  \frac{\partial f}{\partial u^{(n-1)}} (y^{(n-1)}_2-y^{(n-1)}_1)$\\
	Залишилось зробити оцінку:\\
	$|g(0) - g(1)| = |g'(\xi)| \leq |L_1(y_2-y_1)+\dots+L_{n-1}(y^{(n-1)}_2-y^{(n-1)}_1)| \leq \\ \leq |L_1(y_2-y_1)| +\dots+ |L_{n-1}(y^{(n-1)}_2-y^{(n-1)}_1)| \leq L\left(|y_1-y_2|+\dots+|y^{(n-1)}_2-y^{(n-1)}_1|\right)$ 
\end{proof}
	
	Тепер розглянемо задачу Коші:
	\begin{align*}
	\begin{cases}
	\displaystyle y^{(n)} = f(x,y,y',\cdots,y^{(n-1)})\\
	y(x_0)=y_0\\
	y'(x_0) = y_0'\\
	\cdots\\
	y^{(n-1)}(x_0) = y_0^{(n-1)}
	\end{cases}
	\end{align*}
	де $f: \Pi = I_a \cross \Pi_b \rightarrow \mathbb{R}$\\
	$I_a = [x_0 - a, x_0 + a], \\ \Pi_b = [y_0 -b, y_0+b] \cross [y_0'-b,y_0'+b] \cross \cdots \cross [y_0^{(n-1)}-b, y_0^{(n-1)}+b]$
	
	\begin{theorem}[Теорема Пікара]
	Задана функція $f: \Pi \rightarrow \mathbb{R}$ така, що $f \in C(\Pi)$ та під умовою Ліпшиця відносно $y,y',\cdots,y^{(n-1)}$\\
	Тоді задача Коші містить єдиний розв'язок $y_*(x)$ на інтервалі $\\I_h = [x_0-h, x_0+h]$\\
	\textit{Доведення проводиться аналогічним чином, як з рівнянням першого порядку}
	\end{theorem}
	
	\subsection{Деякі типи рівнянь, що допускають зниження порядку}
	\subsubsection{Рівняння, в якої немає залежності від $y$ в правій частині}
	Тобто в нас рівняння буде такого типу:
	\begin{align*}
	y^{(n)} = f(x,y',y'',\dots, y^{(n-1)})
	\end{align*}
	Для нього проводиться наступна заміна: $z(x) = y'(x)$. Тоді\\
	$y'' = z, \dots, y^{(n)} = z^{(n-1)}$\\
	Отримаємо наступне рівняння:\\
	$z^{(n-1)}=f(x,z,z',\dots,z^{(n-2)})$\\
	Ну а далі як пощастить з типажом рівняння
	\bigskip \\
	Також можемо розглянути рівняння такого типу:
	\begin{align*}
	y^{(n)} = f(x,y^{(k)},y^{(k+1)},\dots, y^{(n-1)})
	\end{align*}
	Для нього проводиться наступна заміна: $z(x) = y^{(k)}(x)$\\
	Тоді отримаємо наступне рівняння:\\
	$z^{(n-k)}=f(x,z,z',\dots,z^{(n-k-1)})$\\
	І т.д.
	
	\begin{example}
 Розв'язати рівняння: $xy^{(4)} - y^{'''}=0$\\
	Заміна: $z = y''' \Rightarrow z' = y^{(4)}$\\
	$xz'-z=0 \Rightarrow \displaystyle \frac{dz}{z} = \frac{dx}{x} \Rightarrow z = C_1x$\\
	$y''' = C_1x \Rightarrow \displaystyle y'' = C_1 \frac{x^2}{2} + C_2 \Rightarrow y' = C_1 \frac{x^3}{3!} + C_2x + C_3 \Rightarrow \\ \Rightarrow y = C_1 \frac{x^4}{4!} + C_2 \frac{x^2}{2!}+ C_3x + C_4$\\
	Але $C_1,C_2,C_3,C_4$ - константи, тому можна записати іншим шляхом:\\
	$y = C_1x^4 + C_2x^2 + C_3x + C_4$
	\end{example}
	
	\subsubsection{Рівняння, в якої немає залежності від $x$ в правій частині}
	Тобто в нас рівняння буде такого типу:
	\begin{align*}
	y^{(n)} = f(y,y',y'',\dots, y^{(n-1)})
	\end{align*}
	Проведемо таку заміну: $y' = p(y)$\\
	Далі рахуємо другі, треті і т.д. похідні, але достатньо часто буде другої:\\
	$y'' = p'(y)y'=p'(y)p(y)$\\
	В результаті чого ми отримаємо рівняння від функції $p(y)$ $(n-1)$-го порядку
	
	\begin{example}
 Розв'язати рівняння: $\displaystyle y'' = \frac{1}{4\sqrt{y}}$\\
	Заміна: $y'=p(y) \Rightarrow y''=p'(y)y'=p'p$\\
	$\displaystyle p'p = \frac{1}{4\sqrt{y}} \Rightarrow \frac{dp}{dy} p = \frac{1}{4\sqrt{y}} \Rightarrow p\,dp = \frac{dy}{4\sqrt{y}} \Rightarrow \frac{p^2}{2} = \frac{\sqrt{y}}{2} + C_1 \Rightarrow \\ p = \pm \sqrt{\sqrt{y}+C_1}$\\
	$\displaystyle y' = \pm \sqrt{\sqrt{y}+C_1} \Rightarrow \frac{dy}{\sqrt{\sqrt{y}+C_1}} = \pm dx \Rightarrow \dots \\ \Rightarrow \frac{4}{3}\sqrt{(\sqrt{y}+C_1)^3} - 4C_1\sqrt{\sqrt{y}+C_1} = C_2 \pm x$
	\end{example}

	
	\subsubsection{Рівняння тип 3}
	Буде нехай таке рівняння:
	\begin{align*}
	y^{(n)} = f(y^{(n-2)})
	\end{align*}
	Проведемо наступну заміну: $z(x) = y^{(n-2)}(x)$\\
	Тоді буде таке рівняння:\\
	$z'' = f(z)$\\
	Домножимо обидві частини на $2z'$ і нехай $\displaystyle \int f(z)\,dz = F(z)$\\
	$2z'z'' = 2z'f(z)$\\
	$((z')^2)' = (2F(z))'$\\
	$(z')^2 = 2F(z) + C_1 \Rightarrow z' = \displaystyle \pm \sqrt{F(z)+C_1}$ І т.д.
	
	\begin{example}
 Розв'язати рівняння: $\varphi'' = -k \sin \varphi$, де $\varphi = \varphi(t)$\\
	$2 \varphi' \varphi'' = -2k \varphi' \sin \varphi$\\
	$((\varphi')^2)' = 2k(\cos \varphi)'$\\
	$(\varphi')^2 = 2k\cos \varphi + C_1$\\
	Нехай $C_1 = 0$ (для спрощення). Тоді:\\
	$\varphi' = \pm \sqrt{2k\cos \varphi} \Rightarrow \displaystyle \frac{d\varphi}{\sqrt{2k \cos \varphi}} = \pm dt \Rightarrow t = \pm \frac{1}{\sqrt{2k}} \int \frac{d\varphi}{\sqrt{cos\varphi}} + C_2$
	\end{example}
	\newpage
	
	\section{Лінійні диференціальні рівняння}
	\begin{definition}
 $n$ називається рівняння наступного вигляду:
	\begin{align*}
	y^{(n)} + a_{n-1}(x)y^{(n-1)}+\dots+a_1(x)y'+a_0(x)y=b(x)
	\end{align*}
	де $a_0, a_1,\dots,a_{n-1},b \in C(I)$, $I \subset \mathbb{R}$\\
	\textbf{Розв'язком} називається функція $\varphi \in C^{n}(J)$, $J \subset I$, якщо вона задовольняє цьому рівнянню\\
	Якщо $b(x) \equiv 0$, то рівняння називається \textbf{однорідним}. В інакшому випадку - \textbf{неоднорідним}
	\end{definition}
	
	\begin{theorem}
 Задача Коші для лінійного диференціального рівняння містить єдиний розв'язок.
 	\end{theorem}
 	
	\begin{proof}
	Отже, є в нас задача Коші:\\
	$\begin{cases}
	y^{(n)} + a_{n-1}(x)y^{(n-1)}+\dots+a_1(x)y'+a_0(x)y=b(x)\\
	y(x_0) = y_0\\
	\cdots\\
	y^{(n-1)}(x_0) = y_0^{(n-1)}
	\end{cases}
	$\\
	Доведення теореми буде посилатись на теорему Пікара в диф. рівнянні порядку $n$\\
	Тому перше рівняння системи перепишемо в іншому вигляді:\\
	$y^{(n)} = b(x) - a_{n-1}(x)y^{(n-1)} - \dots - a_1(x)y' - a_0(x)y$\\
	Знайдемо всі її часткові похідні:\\
	$f'_y = -a_0(x)$, $f'_{y'} = -a_1(x)$, $\dots$, $f'_{y^{(n-1)}} = -a_{n-1}(x)$\\
	Всі вони є неперервними функціями на $\Pi = I_a \cross \Pi_b$, тому що \\$a_0,\dots,a_{n-1} \in C(I), I_a \subset I$, $\Pi_b \subset \mathbb{R}^n$ - довільний паралелепіпед навколо точки умов Коші. Отже, функція під умовою Ліпшиця. А значить, спрацьовує теорема Пікара 
\end{proof}

	Визначимо оператор $L: C^n(I) \rightarrow C(I)$ такий, що:
	\begin{align*}
	(Ly)(x) = y^{(n)} + a_{n-1}(x)y^{(n-1)}+\dots+a_1(x)y'+a_0(x)y
	\end{align*}
	Тобто наше рівняння перепишеться як: $(Ly)(x) = b(x)$\\
	\begin{lemma}
 Множина $C^{k}(I)$ є лінійним простором, для якого:
	\begin{align*}
	(f_1+f_2)(x) = f_1(x) + f_2(x)\\
	(\alpha f)(x) = \alpha f(x), \alpha \in \mathbb{R}
	\end{align*}
	\end{lemma}
	
	\begin{lemma}
 	Оператор $L$ є лінійним
 	\end{lemma}
	
	\begin{proof}
	\begin{remark}
 $Ly \in C(I)$
 \end{remark}
	Доведення безпосередньо за означенням:\\
	$L(y+z) = (y+z)^{(n)} + a_{n-1}(x)(y+z)^{(n-1)}+\dots+a_1(x)(y+z)'+a_0(x)(y+z)= \\ =
	y^{(n)} + z^{(n)} + a_{n-1}(x)y^{(n-1)} + a_{n-1}(x)z^{(n-1)} + \dots+a_1(x)y'+a_1(x)z'+a_0(x)y+a_0(x)z = \\ = Ly + Lz$\\
	$L(\alpha y) = (\alpha y)^{(n)} + a_{n-1}(x)(\alpha y)^{(n-1)}+\dots+a_1(x)(\alpha y)'+a_0(x)(\alpha y)= \\ = \alpha y^{(n)} + \alpha a_{n-1}(x)y^{(n-1)}+\dots+\alpha a_1(x)y'+\alpha a_0(x)y = \alpha Ly$ 
	\end{proof}
	
	\begin{proposition}
 Множина розв'язків утворюють лінійний простор $= \ker L$
	\end{proposition}

	\textit{Випливає з означення ядра}
	
	\begin{corollary}
 Якщо $y_1, \dots, y_n$ - розв'язки, то $y=C_1y_1+\dots+C_n y_n$ - розв'язок також
 \end{corollary}
	Тепер перейдемо до розв'язку рівнянь
	
	\subsection{Однорідне рівняння}
	Спробуємо розв'язати рівняння
	\begin{align*}
	y^{(n)} + a_{n-1}(x)y^{(n-1)}+\dots+a_1(x)y'+a_0(x)y = 0
	\end{align*}
	де $a_0, a_1,\dots,a_{n-1}\in C(I)$, $I \subset \mathbb{R}$\\
	Нагадуємо:\\
	\begin{definition}
 Система функцій $\{f_1,\dots,f_n\} \subset C^k(I)$ називається:\\
	- \textbf{лінійно НЕзалежними}, якщо із $\forall x \in I: c_1f_1(x) + \dots + c_n f_n(x) = 0$ випливає, що $c_1 = \dots = c_n = 0$\\
	- \textbf{лінійно залежними}, якщо $\exists c_1, \dots, c_n: c_1^2 + \dots + c_n^2 \neq 0$, для яких $\forall x \in I: c_1f_1(x) + \dots + c_n f_n(x) = 0$
	\end{definition}
	
	\begin{definition}
 Задана система функцій $\{f_1, \dots, f_n\} \in C^{(n-1)}(I)$\\
	Визначимо \textbf{детермінант Вронського} як функцію $W: I \rightarrow \mathbb{R}:$
	\begin{align*}
W[f_1,f_2,\dots,f_n](x) = 
\begin{vmatrix} 
	f_1(x) &  f_2(x) & \dots & f_n(x) \\ 
	f'_1(x) &  f'_2(x) & \dots & f'_n(x) \\
	\vdots &  \vdots & \ddots & \vdots \\
	f^{(n-1)}_1(x) &  f^{(n-1)}_2(x) & \dots & f^{(n-1)}_n(x) \\ 
\end{vmatrix}
	\end{align*}
	\end{definition}
	
	\begin{example}
 Нехай є система $\{1,x,x^2,\dots,x^{k-1}\}$. Тоді\\
	$\displaystyle W[1,x,x^2,\dots,x^{k-1}](x) = 
	\begin{vmatrix} 
	1 &  x & x^2 & \dots & x^{k-1} \\ 
	0 &  1! & 2x & \dots & (k-1)x^{k-2} \\ 
	0 &  0 & 2! & \dots & (k-1)(k-2)x^{k-3} \\
	\vdots & \vdots & \vdots & \ddots & \vdots \\
	0 &  0 & 0 & \dots & (k-1)! \\
\end{vmatrix} = \\ = 1! 2! \dots (k-1)!$
	\end{example}
	
	\begin{example}[Дуже важливий приклад]
	Нехай є система $\displaystyle \{e^{\lambda_1 x}, e^{\lambda_2 x}, \dots, e^{\lambda_n x}\}$\\
	Тут $\lambda_1, \lambda_2, \dots, \lambda_n \in \mathbb{R} (\mathbb{C})$ та всі різні. Тоді маємо:\\
	$\displaystyle W[e^{\lambda_1 x}, e^{\lambda_2 x}, \dots, e^{\lambda_n x}](x) = 
\begin{vmatrix} 
	e^{\lambda_1 x} &  e^{\lambda_2 x} & \dots & e^{\lambda_n x} \\ 
	\lambda_1 e^{\lambda_1 x} &  \lambda_2 e^{\lambda_2 x} & \dots & \lambda_n e^{\lambda_n x} \\ 
	\vdots &  \vdots & \ddots & \vdots \\
	\lambda_1^{n-1} e^{\lambda_1 x} &  \lambda_2^{n-1} e^{\lambda_2 x} & \dots & \lambda_n^{n-1} e^{\lambda_n x} \\
\end{vmatrix} \boxed{=}$\\
Виносимо $e^{\lambda_1 x}$ з першої колони, $e^{\lambda_2 x}$ з другої колони...\\
$\displaystyle \boxed{=}
e^{\lambda_1 x} e^{\lambda_2 x} \dots e^{\lambda_n x}
\begin{vmatrix} 
	1 &  1 & \dots &  1 \\ 
	\lambda_1 &  \lambda_2 & \dots & \lambda_n \\ 
	\vdots &  \vdots & \ddots & \vdots \\
	\lambda_1^{n-1} &  \lambda_2^{n-1} & \dots & \lambda_n^{n-1} \\
\end{vmatrix} = e^{(\lambda_1+\lambda_2+\dots+\lambda_n)x} D_n(\lambda_1, \lambda_2, \dots, \lambda_n)
$\\
Останній множник - детермінант Вандеморда. Із курсу лінійної алгебри,\\
$\displaystyle D_n(\lambda_1, \lambda_2, \dots, \lambda_n) = \prod_{1 \leq j < i \leq n} (\lambda_i-\lambda_j)$\\
Тому остаточно матимемо:\\
$\displaystyle W[e^{\lambda_1 x}, e^{\lambda_2 x}, \dots, e^{\lambda_n x}](x) = e^{(\lambda_1+\lambda_2+\dots+\lambda_n)x} \prod_{1 \leq j < i \leq n} (\lambda_i-\lambda_j)$
\end{example}

	\begin{proposition}
 Якщо $\{f_1, \dots, f_n\} \subset C^{(n-1)}(I)$ - лінійно залежні над $\mathbb{R}$, то $W[f_1,\dots f_n](x) \equiv 0$
	\end{proposition}

	\begin{proof}
Система - л.з., тобто при $c_1, \dots, c_n$, що не всі нулі: \\$c_1f_1(x) + \cdots + c_n f_n(x) = 0$
Продиференціюємо рівняння $(n-1)$ разів. Тоді отримається система:\\
$\begin{cases}
c_1f_1(x) + c_2f_2(x) + \cdots + c_n f_n(x) = 0 \\
c_1f'_1(x) + c_2f'_2(x) + \cdots + c_n f'_n(x) = 0 \\
\vdots \\
c_1f^{(n-1)}_1(x) + c_2f^{(n-1)}_2(x) + \cdots + c_n f^{(n-1)}_n(x) = 0
\end{cases}
$\\
Знову з курсу лінійної алгебри, система має нетрививіальний розв'язок $\iff$ детермінант коефіцієнтів $= 0$\\
У нас $c_1, c_2 ,\dots, c_n$ - нетривіальні, тому матриця коефіцієнтів\\
$ \displaystyle 
\begin{vmatrix} 
	f_1(x) &  f_2(x) & \dots & f_n(x) \\ 
	f'_1(x) &  f'_2(x) & \dots & f'_n(x) \\
	\vdots &  \vdots & \ddots & \vdots \\
	f^{(n-1)}_1(x) &  f^{(n-1)}_2(x) & \dots & f^{(n-1)}_n(x) \\ 
\end{vmatrix} = W[f_1,f_2,\cdots,f_n](x) \equiv 0
$ 
	\end{proof}

\begin{corollary}
 Якщо $\exists x_0: W[f_1,\dots,f_n](x_0) \neq 0$, то $\{f_1, \dots, f_n\}$ - лінійно НЕзалежна\\
\textit{Тут записано просто обернене твердження}\\
\end{corollary}

Повертаючись до \textbf{Ex. 3.1.3.}, обидва детермінанта Вронського ненулеві. Тому система $\{1,x,\dots,x^{k-1}\}$ та $\displaystyle \{e^{\lambda_1 x}, e^{\lambda_2 x}, \dots, e^{\lambda_n x}\}$ - лінійно НЕзалежні
\begin{remark}
 Якщо $W[f_1,\dots, f_n] \equiv 0$, то $\{f_1, \dots, f_n\}$ \underline{може} бути лінійно НЕзалежною
 \end{remark}

\begin{example}
 Розгянемо систему $\{x^2, x|x|\} \subset C'((-2,2))$\\
$\displaystyle W[x^2, x|x|](x) = 
\begin{vmatrix} 
	x^2 &  x|x| \\
	2x &  2|x| \\ 
\end{vmatrix} = 2x^2|x| - 2x^2|x| \equiv 0$\\
Але за означенням л.н.з.,\\
$\displaystyle \forall x \in (-2,2): C_1x^2 + C_2 x|x| = 0  \overset{x = 1, x = -1}{\Rightarrow}
\begin{cases}
C_1 - C_2 = 0\\
C_1 + C_2 = 0
\end{cases} \Rightarrow C_1 = C_2 = 0
$\\
Тому система - лінійно НЕзалежна
\end{example}

\begin{theorem}
Задані $y_1, \dots, y_n \in \ker L$ - тобто розв'язки нашого однорідного рівняння\\
$\{y_1, \dots, y_n\}$ - л.н.з. над $\mathbb{R} \iff W[y_1, \dots, y_n](x) \not\equiv 0$
\end{theorem}

\begin{proof}
\leftproof Дано: $W[y_1, \dots, y_n](x) \not\equiv 0$. Тоді за \textbf{Crl 3.1.4.} система $\{y_1, \dots, y_n\}$ - л.н.з.\\
\bigskip \\
\rightproof Дано: $\{y_1, \dots, y_n\}$ - л.н.з.\\
!Припустимо, що $\exists x_0 \in I: W[y_1, \dots, y_n](x_0) = 0$. Тоді система рівнянь:
$\begin{cases}
c_1y_1(x_0) + c_2y_2(x_0) + \cdots + c_n y_n(x_0) = 0 \\
c_1y'_1(x_0) + c_2y'_2(x_0) + \cdots + c_n y'_n(x_0) = 0 \\
\vdots \\
c_1y^{(n-1)}_1(x_0) + c_2y^{(n-1)}_2(x_0) + \cdots + c_n y^{(n-1)}_n(x_0) = 0
\end{cases}
$
має нетривіальні розв'язки\\
Нехай $c_1 = c_1^0, \dots, c_n = c_n^0$\\
Розглянемо функцію:\\
$y(x) = c_1^0 y_1(x) + \dots + c_n^0 y_n(x)$\\
Якщо продиференціювати $(n-1)$ раз та всюди $x=x_0$, то можна отримати, що:\\
$y(x_0) = y'(x_0) = \dots = y^{(n-1)}(x_0) = 0$\\
Таким чином отримана задача Коші:\\
$
\begin{cases}
 Ly = 0\\
 y(x_0) = 0 \\
 \vdots \\
 y^{(n-1)}(x_0) = 0
\end{cases}
$\\
Проте зауважимо, що $z(x) \equiv 0$ - також розв'язок задачі Коші. Тому в силу єдиності рішення задачі Коші, маємо, що $y(x) \equiv 0$\\
Отже, для $c_1 = c_1^0, \dots, c_n = c_n^0$ отримали: $c_1^0 y_1(x) + \dots + c_n^0 y_n(x) = 0$. Це є означенням л.з., що суперечить нашим припущенням.\\
Висновок: $W[y_1, \dots, y_n](x) \not\equiv 0$ 
\end{proof}

\begin{theorem}
 Позначу через $y_1, y_2, \dots, y_n$ наступні розв'язки задачі Коші:\\
$y_1:
\begin{cases}
 Ly = 0\\
 y(x_0) = 1 \\
 y'(x_0) = 0 \\
 \vdots \\
 y^{(n-1)}(x_0) = 0
\end{cases}
y_2:
\begin{cases}
 Ly = 0\\
 y(x_0) = 0 \\
 y'(x_0) = 1 \\
 \vdots \\
 y^{(n-1)}(x_0) = 0
\end{cases}
\dots
y_n:
\begin{cases}
 Ly = 0\\
 y(x_0) = 0 \\
 y'(x_0) = 0 \\
 \vdots \\
 y^{(n-1)}(x_0) = 1
\end{cases}
$\\
Тоді $\{y_1, y_2, \dots, y_n\}$ утворюють лінійний базис в просторі розв'язків нашого рівняння
\end{theorem}

\begin{proof}
Базис означає лінійну незалежність та презентацію кожного елементу в лінійну комбінацію\\
Перевіримо на л.н.з.:\\
$\displaystyle W[y_1, y_2, \dots, y_n](x_0) = 
\begin{vmatrix} 
	1 &  0 & \dots & 0 \\ 
	0 &  1 & \dots & 0 \\
	\vdots &  \vdots & \ddots & \vdots \\
	0 &  0 & \dots & 1 \\
\end{vmatrix} = 1 \neq 0$\\Отже, за \textbf{Crl. 3.1.4.}, система - л.н.з.
\\
Доведемо, що будь-який розв'язок $y = y_1c_1 + \dots + y_n c_n$. Коротше, треба знайти $c_1, \dots, c_n$ і довести, що вони єдині\\
Диференціюємо $(n-1)$ разів і вставляємо $x_0$ кожного разу. Тоді\\
$c_1 = y(x_0), \dots, c_n=y^{(n-1)}(x_0)$. Ці константи виражаються єдиним чином\\
Отже, $\exists! c_1,\dots,c_n: y = c_1 y_1 + \dots + c_n y_n$\\
Висновок: $\{y_1, \dots. y_n\}$ - базис 
\end{proof}

\begin{corollary}
 $\dim L = n$
 \end{corollary}

\begin{corollary}
 Будь-яка л.н.з. система з $n$ розв'язків утворює базис в просторі його рішень
 \end{corollary}

\begin{definition}
 Лінійним базисом $\{y_1 \dots, y_n\}$ в просторі розв'язків рівняння називають \textbf{фундаментальною системою розв'язків}\\
В той же час $y = c_1 y_1 + \dots + c_n y_n$ - \textbf{загальний розв'язок}\\
\end{definition}

\begin{lemma}
 Якщо $a_{ij} \in C'(I) i,j = \overline{1,n}$, то\\
$\displaystyle \left(
\begin{vmatrix} 
	a_{11}(x) &  a_{12}(x) & \dots & a_{1n}(x) \\ 
	a_{21}(x) &  a_{22}(x) & \dots & a_{2n}(x) \\ 
	\vdots &  \vdots & \ddots & \vdots \\
	a_{n1}(x) &  a_{n2}(x) & \dots & a_{nn}(x) \\ 
\end{vmatrix}
\right)' = \\ =
\begin{vmatrix} 
	a'_{11}(x) &  a'_{12}(x) & \dots & a'_{1n}(x) \\ 
	a_{21}(x) &  a_{22}(x) & \dots & a_{2n}(x) \\ 
	\vdots &  \vdots & \ddots & \vdots \\
	a_{n1}(x) &  a_{n2}(x) & \dots & a_{nn}(x) \\ 
\end{vmatrix}
+
\begin{vmatrix} 
	a_{11}(x) &  a_{12}(x) & \dots & a_{1n}(x) \\ 
	a'_{21}(x) &  a'_{22}(x) & \dots & a'_{2n}(x) \\ 
	\vdots &  \vdots & \ddots & \vdots \\
	a_{n1}(x) &  a_{n2}(x) & \dots & a_{nn}(x) \\ 
\end{vmatrix}
+ \\ \\ + \dots +
\begin{vmatrix} 
	a_{11}(x) &  a_{12}(x) & \dots & a_{1n}(x) \\ 
	a_{21}(x) &  a_{22}(x) & \dots & a_{2n}(x) \\ 
	\vdots &  \vdots & \ddots & \vdots \\
	a'_{n1}(x) &  a'_{n2}(x) & \dots & a'_{nn}(x) \\ 
\end{vmatrix}
$
\end{lemma}

\begin{proof}
Доведення проводимо за означенням:\\
$\displaystyle \begin{vmatrix} 
	a_{11}(x) &  a_{12}(x) & \dots & a_{1n}(x) \\ 
	a_{21}(x) &  a_{22}(x) & \dots & a_{2n}(x) \\ 
	\vdots &  \vdots & \ddots & \vdots \\
	a_{n1}(x) &  a_{n2}(x) & \dots & a_{nn}(x) \\ 
\end{vmatrix} = \sum_{\sigma \in S_n} \epsilon_\sigma a_{1 \sigma(1)}(x) \cdot a_{2 \sigma(2)}(x) \cdot \hdots \cdot a_{n \sigma(n)}(x)$\\
Тут $S_n$ - група перестановок множини $A_n = \{1,2,\dots,n\}$\\
$\sigma(k) \in A_n$ - значення перестановки $\sigma$ на елементі $k \in A_n$\\
$ \epsilon_\sigma = 
\begin{cases}
1, \textrm{парна перестановка}\\
-1, \textrm{непарна перестановка}
\end{cases}
$\\
А тепер візьмемо похідну від правої частини:\\
$\displaystyle \left(\sum_{\sigma \in S_n} \epsilon_\sigma a_{1 \sigma(1)}(x) \cdot a_{2 \sigma(2)}(x) \cdot \hdots \cdot a_{n \sigma(n)}(x)\right)' = \\ \sum_{\sigma \in S_n} \epsilon_\sigma a'_{1 \sigma(1)}(x) \cdot a_{2 \sigma(2)}(x) \cdot \hdots \cdot a_{n \sigma(n)}(x) + \\ + \sum_{\sigma \in S_n} \epsilon_\sigma a_{1 \sigma(1)}(x) \cdot a'_{2 \sigma(2)}(x) \cdot \hdots \cdot a_{n \sigma(n)}(x) + \\ + \dots + \sum_{\sigma \in S_n} \epsilon_\sigma a_{1 \sigma(1)}(x) \cdot a_{2 \sigma(2)}(x) \cdot \hdots \cdot a'_{n \sigma(n)}(x) \overset{def}{=} \\
\\
=\begin{vmatrix} 
	a'_{11}(x) &  a'_{12}(x) & \dots & a'_{1n}(x) \\ 
	a_{21}(x) &  a_{22}(x) & \dots & a_{2n}(x) \\ 
	\vdots &  \vdots & \ddots & \vdots \\
	a_{n1}(x) &  a_{n2}(x) & \dots & a_{nn}(x) \\ 
\end{vmatrix}
+
\begin{vmatrix} 
	a_{11}(x) &  a_{12}(x) & \dots & a_{1n}(x) \\ 
	a'_{21}(x) &  a'_{22}(x) & \dots & a'_{2n}(x) \\ 
	\vdots &  \vdots & \ddots & \vdots \\
	a_{n1}(x) &  a_{n2}(x) & \dots & a_{nn}(x) \\ 
\end{vmatrix}
+ \\ \\ + \dots +
\begin{vmatrix} 
	a_{11}(x) &  a_{12}(x) & \dots & a_{1n}(x) \\ 
	a_{21}(x) &  a_{22}(x) & \dots & a_{2n}(x) \\ 
	\vdots &  \vdots & \ddots & \vdots \\
	a'_{n1}(x) &  a'_{n2}(x) & \dots & a'_{nn}(x) \\ 
\end{vmatrix}
$ 
\end{proof}

\begin{theorem}[Формула Остроградського-Якобі]
Задана $\{y_1,\dots, y_n\}$ - фундаментальна система розв'язків нашого рівняння та якась т. $x_0 \in I$. Тоді
\begin{align*}
W[y_1, \dots, y_n](x) = W[y_1, \dots, y_n](x_0)e^{\displaystyle -\int_{x_0}^x a_{n-1}(t)\,dt}
\end{align*}
\end{theorem}

\begin{proof}
Знайдемо похідну від детермінанта Вронського (\textbf{Lm. 3.1.9.}):\\
$\displaystyle W'[y_1, \dots, y_n](x) = 
\begin{vmatrix} 
	y'_1(x) &  y'_2(x) & \dots & y'_n(x) \\ 
	y'_1(x) &  y'_2(x) & \dots & y'_n(x) \\
	\vdots &  \vdots & \ddots & \vdots \\
	y^{(n-1)}_1(x) &  y^{(n-1)}_2(x) & \dots & y^{(n-1)}_n(x) \\ 
\end{vmatrix}
+ \\ +
\begin{vmatrix} 
	y_1(x) &  y_2(x) & \dots & y_n(x) \\ 
	y''_1(x) &  y''_2(x) & \dots & y''_n(x) \\
	y''_1(x) &  y''_2(x) & \dots & y''_n(x) \\
	\vdots &  \vdots & \ddots & \vdots \\
	y^{(n-1)}_1(x) &  y^{(n-1)}_2(x) & \dots & y^{(n-1)}_n(x) \\ 
\end{vmatrix}
+ \dots +
\begin{vmatrix} 
	y_1(x) &  y_2(x) & \dots & y_n(x) \\ 
	y'_1(x) &  y'_2(x) & \dots & y'_n(x) \\
	\vdots &  \vdots & \ddots & \vdots \\
	y^{(n-2)}_1(x) &  y^{(n-2)}_2(x) & \dots & y^{(n-2)}_n(x) \\ 
	y^{(n)}_1(x) &  y^{(n)}_2(x) & \dots & y^{(n)}_n(x) \\ 
\end{vmatrix} =
$\\
Тут всі детермінанти, окрім останнього, онуляться через однакові рядки\\
\\
$\displaystyle = \begin{vmatrix} 
	y_1(x) &  y_2(x) & \dots & y_n(x) \\ 
	y'_1(x) &  y'_2(x) & \dots & y'_n(x) \\
	\vdots &  \vdots & \ddots & \vdots \\
	y^{(n-2)}_1(x) &  y^{(n-2)}_2(x) & \dots & y^{(n-2)}_n(x) \\ 
	y^{(n)}_1(x) &  y^{(n)}_2(x) & \dots & y^{(n)}_n(x) \\ 
\end{vmatrix} \boxed{=}$\\
Оскільки $y_1, \dots, y_n$ - розв'язки, то ми можемо виразити старші подіхні:\\
$y^{(n)}_i = -a_0(x)y_j - a_1(x)y_j' -\dots - a_{n-1}(x)y_j^{(n-1)}, j=\overline{1,n}$\\
Підставимо в наш детермінант і зробимо такі кроки:\\
- помножимо перший рядок на $a_0$ і додамо до останнього рядка;\\
- помножимо другий рядок на $a_1$ і додамо до останнього рядка;\\
...\\
Нарешті, винесемо $-a_{n-1}$. Отримаємо:\\
$\displaystyle \boxed{=} -a_{n-1}(x)\begin{vmatrix} 
	y_1(x) &  y_2(x) & \dots & y_n(x) \\ 
	y'_1(x) &  y'_2(x) & \dots & y'_n(x) \\
	\vdots &  \vdots & \ddots & \vdots \\
	y^{(n-1)}_1(x) &  y^{(n-1)}_2(x) & \dots & y^{(n-1)}_n(x) \\ 
\end{vmatrix} = -a_{n-1}(x)W[y_1, \dots, y_n](x)$\\
Отже, отримали таку тотожність:\\
$W'[y_1, \dots, y_n](x) = -a_{n-1}(x)W[y_1, \dots, y_n](x)$\\
А це - диф. рівняння з відокремленими змінними, яку ми розв'яжемо:\\
$\displaystyle \frac{dW[y_1, \dots, y_n](t)}{dt} = -a_{n-1}(t)$\\
$\displaystyle \frac{dW[y_1, \dots, y_n](t)}{W[y_1, \dots, y_n](t)} = -a_{n-1}(t)\,dt$\\
Інтегруємо на інтервалі $[x,x_0]$:\\
$\displaystyle \ln \abs{\frac{W[y_1, \dots, y_n](x)}{W[y_1, \dots, y_n](x_0)}} = - \int_{x_0}^x a_{n-1}(t)\,dt$\\
$\displaystyle W[y_1, \dots, y_n](x) = W[y_1, \dots, y_n](x_0)e^{\displaystyle -\int_{x_0}^x a_{n-1}(t)\,dt}$\\
Взагалі, тут мав би бути знак $\pm$, але якщо підставити $x=x_0$, то залишиться лише $ +$ 
\end{proof}

Метод розв'язку лінійних однорідних диференціальних рівнянь найчастіше другого порядку саме базується на теоремі Остроградського-Якобі. Спочатку ми вгадуємо перший частковий розв'язок, а далі за формулою шукаємо другий частковий, а згодом можна отримати загальний розв'язок
\\
\\
\textbf{Example} Розв'язати рівняння: $(2x+1)y'' + 4xy' - 4y = 0$\\
Буду розглядувати на інтервалі $\displaystyle x>-\frac{1}{2}$\\
$\displaystyle y'' + \frac{4x}{2x+1}y' - \frac{4}{2x+1}y = 0$\\
Можна вгадати, що $y_1 = x$ - частковий розв'язок\\
Тоді за формулою Остроградського-Якобі:\\
$\displaystyle \begin{vmatrix}
x & y_2 \\
1 & y_2' \\
\end{vmatrix}
= W_0 e^{-\int_{1}^x \frac{4t}{2t+1}\,dt}$. Тут $x_0 = 1$\\
$\displaystyle -\int_{1}^x \frac{4t}{2t+1}\,dt = \dots = -2x + \ln(2x+1) + 2 - \ln3$\\
$\Rightarrow W_0 e^{-\int_{1}^x \frac{4t}{2t+1}\,dt} = W_0 e^{\ln(2x+1)} e^{-2x} e^{2 - \ln3} = W_1(2x+1)e^{-2x}$\\
Таким чином за нашою формулою:\\
$xy_2'-y_2=W_1(2x+1)e^{-2x}$\\
Ну а тут стандартне диф. рівняння першого порядку. Поділимо на $x^2$ і зауважимо:\\
$\displaystyle \frac{y_2'x-y_2}{x^2} = W_1 \frac{2^{-2x}2x + e^{-2x}}{x^2} \Rightarrow \left(\frac{y_2}{x}\right)' = -W_1 \left( \frac{e^{-2x}}{x} \right)'$\\
$\displaystyle \frac{y_2}{x} = -W_1 \frac{e^{-2x}}{x}$\\
$y_2 = -W_1e^{-2x}$\\
Отже, остаточно загальний розв'язок:\\
$y = C_1x + C_2e^{-2x}$
\bigskip \\

\subsection{Неоднорідне рівняння}
Спробуємо розв'язати рівняння
\begin{align*}
y^{(n)} + a_{n-1}(x)y^{(n-1)}+\dots+a_1(x)y'+a_0(x)y = b(x)
\end{align*}
де $a_0, a_1,\dots,a_{n-1}, b\in C(I)$, $I \subset \mathbb{R}$

\begin{theorem}[Про структуру розв'язків]
$y$ - розв'язок неоднорідного рівняння $\iff y = y_{g.h.} + y_{p.inh.}$\\
Де $y_{g.h.}$ - загальний розв'язок однорідного рівняння, а $y_{p.inh.}$ - частковий розв'язок неоднорідного рівняння
\end{theorem}

\begin{proof}
\rightproof Дано: $y$ - розв'язок неоднорідного рівняння\\
Розглянемо $y_0 = y - y_{p.inh.}$. Для цього маємо:\\
$Ly_0 = L(y - y_{p.inh,}) = Ly - Ly_{p.inh.} = b(x) - b(x) = 0$\\
$\Rightarrow y_0 = y_{g.h} \Rightarrow y = y_{g.h.} + y_{p.inh.}$\\
\\
\leftproof Дано: $y = y_{g.h.} + y_{p.inh.}$\\
Тоді $Ly = L(y_{g.h.} + y_{p.inh.}) = Ly_{g.h.} + Ly_{p.inh.} = b(x) \Rightarrow y$ - розв'язок неоднорідного рівняння 
\end{proof}

\begin{corollary}
Якщо $y_1$ - розв'язок $Ly_1 = b_1(x)$, а $y_2$ - розв'язок $Ly_2 = b_2(x)$, то\\
$y = \beta_1 y_1 + \beta_2 y_2$ - розв'язок $Ly = \beta_1 b_1(x) + \beta_2 b_2(x)$
\end{corollary}

Для неоднорідних рівнянь існує поки єдиний загальних вихід, як розв'язати рівняння\\
\textbf{Метод варіації довільних сталих}\\
Спочатку знайдемо $y_{g.h.}$ з нашого рівняння, тобто:\\
$y^{(n)} + a_{n-1}(x)y^{(n-1)}+\dots+a_1(x)y'+a_0(x)y = 0$\\
Якщо вважати, що $y_1, \dots, y_n$ - фундаментальна система розв'язків, то\\
$y_{g.h.} = c_1y_1 + \dots + c_n y_n$\\
\\
Наш розв'язок ми будемо шукати в такому вигляді:\\
$y = c_1(x)y_1 + \dots + c_n(x)y_n$\\
А тут $c_1(x), \dots, c_n(x)$ - такі функції, задовільняючи наступним умовам:\\
$\begin{cases}
c'_1(x)y_1 + c'_2(x)y_2 \dots + c'_n(x)y_n = 0 \\
c'_1(x)y'_1 + c'_2(x)y'_2 \dots + c'_n(x)y'_n = 0 \\
\vdots \\
c'_1(x)y^{(n-2)}_1(x) + c'_2(x)y^{(n-2)}_2(x) \dots + c'_n(x)y^{(n-2)}_n(x) = 0 \\
\end{cases}
$\\
Використовуючи всі наші умови, ми підставимо наше $y$ до неоднорідного рівняння. Але перед цим знайдемо похідні:\\
$y' = c'_1(x)y_1(x) + c_1(x)y'_1(x) + \dots + c'_n(x)y_n(x) + c_n(x)y'_n(x) \overset{\textrm{умова}}{=} \\ = c_1(x) y'_1(x) + \dots + c_n(x) y'_n(x)$\\
$y'' = c'_1(x)y'_1(x) + c_1(x)y''_1(x) + \dots + c'_n(x)y'_n(x) + c_n(x)y''_n(x) \overset{\textrm{умова}}{=} \\ = c_1(x) y''_1(x) + \dots + c_n(x) y''_n(x)$\\
...\\
$y^{(n-1)} = c'_1(x)y^{(n-2)}_1(x) + c_1(x)y^{(n-1)}_1(x) + \dots + c'_n(x)y^{(n-2)}_n(x) + c_n(x)y^{(n-1)}_n(x) \overset{\textrm{умова}}{=} \\ = c_1(x) y^{(n-1)}_1(x) + \dots + c_n(x) y^{(n-1)}_n(x)$\\
Легко побачити, що завдяки системі зверху, ми можемо функції $c_1, \dots, c_n$ сприйняти як константу, що виноситься з похідної\\
$y^{n} = c'_1(x)y^{(n-1)}_1(x) + c_1(x)y^{(n)}_1(x) + \dots + c'_n(x)y^{(n-1)}_n(x) + c_n(x)y^{(n)}_n(x)$\\
Підставляємо в неоднорідне рівняння:\\
$\left(c'_1(x)y^{(n-1)}_1(x) + \cdots + c'_n(x)y^{(n-1)}_n(x) + c_1(x)y^{(n)}_1(x) + \cdots + c_n(x)y^{(n)}_n(x) \right) + \\ + a_{n-1}(x)\left(c_1(x) y^{(n-1)}_1(x) + \dots + c_n(x) y^{(n-1)}_n(x)\right) + \dots + \\ + a_1(x)\left(c_1(x) y'_1(x) + \dots + c_n(x) y'_n(x)\right) + \\ + a_0(x)\left(c_1(x)y_1(x) + \dots + c_n(x)y_n(x)\right) = b(x)$\\
І перегрупуємо ці доданки:\\
$c_1(x)\left(y^{(n)}_1(x) + a_{n-1}(x)y^{(n-1)}_1(x) + \dots + a_1(x)y'_1(x) +a_0(x)y_1(x) \right) + \\ + c_2(x)\left(y^{(n)}_2(x) + a_{n-1}(x)y^{(n-1)}_2(x) + \dots + a_1(x)y'_2(x) +a_0(x)y_2(x) \right) + \dots + \\ + c_n(x)\left(y^{(n)}_n(x) + a_{n-1}(x)y^{(n-1)}_n(x) + \dots + a_1(x)y'_n(x) +a_0(x)y_n(x) \right) + \\ +
c'_1(x)y_1^{(n-1)}(x) + \dots + c'_n(x)y_n^{(n-1)}(x)
= b(x)$\\
Але оскільки $y_1, \dots, y_n$ - фундаментальна система розв'язків, тобто \\ $Ly_i = 0$, $i = \overline{1,n}$, то залишається:\\
$c'_1(x)y_1^{(n-1)}(x) + \dots + c'_n(x)y_n^{(n-1)}(x) = b(x)$\\
Дане рівняння додамо до нашої системи.\\
\\
В результаті\\
$\begin{cases}
c'_1(x)y_1 + c'_2(x)y_2 \dots + c'_n(x)y_n = 0 \\
c'_1(x)y'_1 + c'_2(x)y'_2 \dots + c'_n(x)y'_n = 0 \\
\vdots \\
c'_1(x)y^{(n-2)}_1(x) + c'_2(x)y^{(n-2)}_2(x) \dots + c'_n(x)y^{(n-2)}_n(x) = 0 \\
c'_1(x)y_1^{(n-1)}(x) + \dots + c'_n(x)y_n^{(n-1)}(x) = b(x)
\end{cases}$\\
Розв'язуючи відосно $c'_1(x) \dots, c'_n(x)$, отримаємо, що вона має єдиний розв'язок, оскільки детермінант матриці коефіцієнтів - детермінант Вронського - ненулевий, згідно з тим, що $y_1, \dots, y_n$ - фундаментальна система\\
Залишилось їх проінтегрувати:\\
$\displaystyle c_1(x) = \int_{x_0}^x c'_1(t)\,dt + \tilde{c_1}, \dots, c_n(x) = \int_{x_0}^x c'_n(t)\,dt + \tilde{c_n}$\\
Та підставити в наш початковий $y$:\\
$\displaystyle y = \left(\int_{x_0}^x c'_1(t)\,dt\right)y_1 + \dots + \left(\int_{x_0}^x c'_n(t)\,dt\right)y_n + \tilde{c_1}y_1 + \dots + \tilde{c_n}y_n = \\ = y_{p.inh.} + y_{g.h.} \iff y$ - розв'язок нашого неоднорідного рівняння\\
\\
Отже, враховуючи всі наші умови:\\ $y = c_1(x)y_1 + \dots + c_n(x)y_n$ - розв'язок лінійного неоднорідного рівняння

\begin{example}
 Розв'язати рівняння: $\displaystyle y'' + \frac{x}{1-x}y' - \frac{1}{1-x}y = x-1$\\
Тут фундаментальна система розв'язків: $y_1 = e^x$, $y_2 = x$\\
Запишемо загальний розв'язок:\\
$y = c_1(x)e^x +c_2(x)x$\\
Тут $c_1(x), c_2(x)$ - такі функції, задовільняючи умовам:\\
$\begin{cases}
c'_1(x)e^x + c'_2(x)x = 0\\
c'_1(x)e^x + c'_2(x) = x-1
\end{cases}$\\
Розв'язки системи: $c'_1(x) = xe^{-x}$, $c'_2(x) = -1$\\
$\Rightarrow c_1(x) = \dots = -xe^{-x} -e^{-x} + \tilde{c_1}$\\
$\Rightarrow c_2(x) = -x + \tilde{c_2}$\\
Отже,\\
$y = e^x(-xe^{-x}-e^{-x}+\tilde{c_1}) + x(-x+\tilde{c_2}) = -x - 1 + \tilde{c_1}e^x - x^2 + x \tilde{c_2}$\\
$\Rightarrow y = \tilde{c_1}e^{x} + \tilde{c_2}x - (x^2 + 1)$
\end{example}

\subsection{Однорідне рівняння з постійними коефіцієнтами}
Спробуємо розв'язати рівняння
\begin{align*}
y^{(n)} + a_{n-1}y^{(n-1)}+\dots+a_1y'+a_0y = 0
\end{align*}
де $a_0, a_1,\dots,a_{n-1} \in \mathbb{R}$\\
\\ 
\textbf{Деяка інформація про комплекснозначні функції в полі дійсних чисел}\\
$f: \mathbb{R} \rightarrow \mathbb{C}:$\\
$f(x) = u(x) + iv(x)$, де $u(x) = \Re f(x), v(x) = \Im f(x)$\\
Похідна від цієї функції визначається таким чином:\\
$f'(x) = u'(x) + iv'(x)$\\
Всі властивості похідних зберігаються для комплекснозначних функцій
\\
\\
Визначимо ще один оператор $D$ - оператор диференціювання:\\
$Df = f'$\\
Тобто наше рівняння матиме вигляд:\\
$Ly = D^n y + a_{n-1}D^{n-1} y + \dots + a_1Dy + a_0Iy = 0$\\
Розглянемо функцію $\displaystyle y = e^{\lambda x}, \lambda \in \mathbb{C}$. Підставимо в наше рівняння:\\
$Le^{\lambda x} = \lambda^n e^{\lambda x} + a_{n-1} \lambda^{n-1} e^{\lambda x} + \dots + a_1 \lambda e^{\lambda x} + a_0 e^{\lambda x} = \\ = e^{\lambda x} \left( \lambda^n + a_{n-1} \lambda^{n-1} + \dots + a_1 \lambda + a_0 \right) = e^{\lambda x} P(\lambda) = 0$\\
$\Rightarrow Le^{\lambda x} = e^{\lambda x} P(\lambda) = 0 \iff P(\lambda) = 0$\\
\begin{definition}
 \textbf{Характеристичним многочленом} будемо називати вираз:
\begin{align*}
P(\lambda) = \lambda^n + a_{n-1} \lambda^{n-1} + \dots + a_1 \lambda + a_0
\end{align*}
\end{definition}
З наших міркувань отримали, що:\\
	\begin{proposition}
 $\displaystyle e^{\lambda x}$ - корінь рівняння $\iff$ $\lambda \in \mathbb{C}$ - корінь характеристичного полінома $P(\lambda) = 0$
	\end{proposition}

\textbf{I. Випадок різних (лише дійсних) коренів}
\begin{theorem}
 Система $\displaystyle \{e^{\mu_1 x}, \dots, e^{\mu_n x}\}$ є фундаментальною системою розв'язків. Причому $\mu_1, \dots, \mu_n \in \mathbb{R}$ - різні корені характеристичного полінома
\end{theorem}

\begin{proof}
Згідно з \textbf{Ex. 3.1.3.}, така система є лінійно НЕзалежною. Оскільки $\mu_1, \dots, \mu_n \in \mathbb{R}$ - різні корені характеристичного полінома, то $e^{\mu_1 x}, \dots, e^{\mu_n x}$ - розв'язки нашого рівняння. Отже, за \textbf{Crl. 3.1.7.(2)}, система є фундаментальною 
\end{proof}

\textbf{Example 1.} Розв'язати рівняння: $y'' - y = 0$\\
Запишемо характеристичний поліном:\\
$P(\lambda) = \lambda^2 - 1 = 0$\\
Звідси $\lambda_1 = 1, \lambda_2 = -1 \overset{Th. 3.3.I.}{\Rightarrow} \{e^x, e^{-x}\}$ - фундаментальна система.
Отже, $y = C_1e^x + C_2e^{-x}$
\\
\bigskip \\
\textbf{II. Випадок різних (можливо комплексних) коренів}
\begin{theorem}
 Система $\\ \displaystyle \{e^{\mu_1 x}, \dots, e^{\mu_l x}\} \cup \{e^{\alpha_1 x} \cos \omega_1 x, e^{\alpha_1 x} \sin \omega_1 x, \dots, e^{\alpha_k x} \cos \omega_k x, e^{\alpha_k x} \sin \omega_k x\}$ \\ є фундаментальною системою розв'язків. Причому $\mu_1, \dots, \mu_l \in \mathbb{R}$ та $\lambda_1 = \alpha_1 + i\omega_1, \dots, \lambda_k = \alpha_k + i\omega_k \in \mathbb{C}$ - різні корені характеристичного полінома
\end{theorem}

\begin{proof}
До речі, якщо $\lambda_1, \dots, \lambda_k$ - корні характеристичного полінома, то (курс ліналу) $\overline{\lambda_1}, \dots, \overline{\lambda_k}$ - також є коренями\\
Отже, за умовою і \textbf{Prp. 3.3.2.}, фундаментальний розв'язок рівняння задається системою\\
$\{e^{\mu_1 x}, \dots, e^{\mu_l x}, e^{\lambda_1 x}, e^{\overline{\lambda_1} x}, \dots, e^{\lambda_k x}, e^{\overline{\lambda_k} x}\}$\\
Замінимо цю систему функцій наступною системою:\\
$\displaystyle \{e^{\mu_1 x}, \dots, e^{\mu_l x}, \frac{e^{\lambda_1 x} + e^{\overline{\lambda_1} x}}{2}, \frac{e^{\lambda_1 x} - e^{\overline{\lambda_1} x}}{2i}, \dots, \frac{e^{\lambda_k x} + e^{\overline{\lambda_k} x}}{2}, \frac{e^{\lambda_k x} - e^{\overline{\lambda_k} x}}{2i}\}$\\
Маємо права, оскільки від цього л.н.з. система не зміниться. В силу лінійності оператора $L$, вони теж будуть розв'зяками. Більш того, за формулами Ейлера:\\
$\displaystyle \frac{e^{\lambda_j x} + e^{\overline{\lambda_j} x}}{2} = e^{\alpha_j x} \cos \omega_j x$\\
$\displaystyle \frac{e^{\lambda_j x} - e^{\overline{\lambda_j} x}}{2i} = e^{\alpha_j x} \sin \omega_j x$\\
Отримали, отримаємо бажану систему:\\
$\{e^{\mu_1 x}, \dots, e^{\mu_l x}\} \cup \{e^{\alpha_1 x} \cos \omega_1 x, e^{\alpha_1 x} \sin \omega_1 x, \dots, e^{\alpha_k x} \cos \omega_k x, e^{\alpha_k x} \sin \omega_k x\}$ 
\end{proof}

\begin{remark}
 Тут всього $n$ функцій в системі: $l$ - дійсних і $2k$ - копмлексних. $l + 2k = n$
 \end{remark}
 
\begin{example}
Розв'язати рівняння: $y'' + y = 0$\\
Запишемо характеристичний поліном:\\
$P(\lambda) = \lambda^2 + 1 = 0$\\
Звідси $\lambda_1 = i, \lambda_2 = -i \overset{Th. 3.3.II.}{\Rightarrow} \{\cos x, \sin x\}$ - фундаментальна система.
Отже, $y = C_1 \cos x + C_2 \sin x$
\end{example}

\textbf{III. Випадок кратних (можливо комплексних) коренів}
\begin{theorem}
 Система $\\ \displaystyle \bigcup_{j=1}^l \{e^{\mu_j x}, x e^{\mu_j x}, \dots, x^{s_j-1}e^{\mu_j x}\} \cup \\ \cup \displaystyle \bigcup_{j=1}^k \{e^{\alpha_j x} \cos \omega_j x, e^{\alpha_j x} \sin \omega_j x, \dots, x^{r_j-1}e^{\alpha_j x} \cos \omega_j x, x^{r_j-1}e^{\alpha_j x} \sin \omega_j x\}$ \\ є фундаментальною системою розв'язків. Причому $\mu_1, \dots, \mu_l \in \mathbb{R}$ з кратністю $s_1, \dots, s_l$ та $\lambda_1 = \alpha_1 + i\omega_1, \dots, \lambda_k = \alpha_k + i\omega_k \in \mathbb{C}$ з кратністю $r_1, \dots, r_k$- різні корені характеристичного полінома
\end{theorem}

\begin{proof}
Доведення дуже масивне. Тому розіб'ємо на 3 леми:

\begin{lemma}
 Якщо $\lambda_j \in \mathbb{C}$ - корінь кратності $s_j$ характеристичного полінома $P{\lambda} = 0$, то розв'язком нашого рівняння буде:\\
$\displaystyle y = x^p e^{\lambda_j x}, \forall p \in \mathbb{N}: 0 \leq p < s_j$
\end{lemma}

\begin{proof}
Зазначимо, що справедлива така рівність:\\
$\displaystyle  D^p_{\lambda} {e^{\lambda x}} \Big|_{\lambda = \lambda_j} = \left(\frac{d}{d \lambda}\right)^p {e^{\lambda x}} \Big|_{\lambda = \lambda_j} = x^p e^{\lambda_j x}$\\
Отриманою функцією подіємо на оператор:\\
$L\left(D^p_{\lambda} {e^{\lambda x}} \Big|_{\lambda = \lambda_j} \right) = \left(D^n_x + a_{n-1}D^{n-1}_x+\dots+a_1D_x +a_0I_x \right) \left(D^p_{\lambda} {e^{\lambda x}} \Big|_{\lambda = \lambda_j} \right) = $\\
$D^a_\lambda D^b_x = D^b_x D^a_\lambda$\\
$\displaystyle = D^p_{\lambda} \left(P(\lambda) e^{\lambda x} \right) \Big|_{\lambda = \lambda_j} \overset{\textrm{пр. Лейбниця}}{=} \sum_{q=0}^p C_p^q P^{(q)}(\lambda) (e^{\lambda x})^{(p-q)} \Big|_{\lambda = \lambda_j} = \sum_{q=0}^p C_p^q P^{(q)}(\lambda_j) \lambda_j^{p-q} e^{\lambda_j x}$\\
Оскільки $\lambda_j$ - корінь кратності $s_j$, то з курсу ліналу,\\
$P(\lambda_j) = \dots = P^{(s_j - 1)}(\lambda_j) = 0$, але $P^{(s_j)}(\lambda_j) \neq 0$\\
Тому при $0 \leq p < s_j$ отримаємо, що:\\
$\displaystyle \sum_{q=0}^p C_p^q P^{(q)}(\lambda_j) = 0 \Rightarrow L(x^p e^{\lambda_j x}) = 0$\\
Отже, $x^p e^{\lambda_j x}$ - розв'язки 
\end{proof}

\begin{lemma}
 Система $\{e^{\lambda x}, x e^{\lambda x}, \dots, x^{s-1} e^{\lambda x}\}$ є лінійно НЕзалежною над $\mathbb{C}$ (для довільних $j$ в нашому випадку)
 \end{lemma}

\begin{proof}
Припустимо, що ця система - л.з., тобто:\\
$\exists C_0, \dots, C_{s-1} \in \mathbb{C}$ нетривіальні: $C_0e^{\lambda x} + C_1 xe^{\lambda x} + \dots + C_{s-1} x^{s-1}e^{\lambda x} = 0$\\
Звідси випливає, що:\\
$e^{\lambda x} (C_0 + C_1x + \dots + C_{s-1}x^{s-1}) = 0$\\
Отримаємо, що права дужка має бути нулевою. Це означає, що система $\{1, x, \dots, x^{s-1}\}$ - л.з., що суперечить (в силу \textbf{Ex. 3.1.3.(1)}).\\
Тому $\{e^{\lambda x}, x e^{\lambda x}, \dots, x^{s-1} e^{\lambda x}\}$ - л.н.з. 
\end{proof}

\begin{lemma}
 Якщо $\lambda_1, \dots, \lambda_m \in \mathbb{C}$ - різні комплексні числа з кратністю $s_1, \dots, s_m$, тоді система\\
$\{e^{\lambda_1 x}, x e^{\lambda_1 x}, \dots, x^{s_1-1} e^{\lambda_1 x}\} \cup \dots \cup \{e^{\lambda_m x}, x e^{\lambda_m x}, \dots, x^{s_m-1} e^{\lambda_m x}\}$ є лінійно НЕзалежною над $\mathbb{C}$
\end{lemma}

\begin{proof}
Знову припустимо, що ля система - л.з., тобто:\\
$\exists C_{pq} \in \mathbb{C}$ не всі нулі: $\displaystyle \sum_{p=1}^m \sum_{q=0}^{s_p-1} C_{pq}x^qe^{\lambda_p x} = 0$\\
Перепозначу $\displaystyle \sum_{q=0}^{s_p - 1} C_{pq}x^q = f_p(x)$. Тоді\\
$f_1(x)e^{\lambda_1 x} + f_2(x)e^{\lambda_2 x} + \dots + f_m(x)e^{\lambda_m x} = 0$\\
Через те, що не всі $C_{pq}$ нулеві, то принаймні одна з $f_p(x)$ є ненулевою. Вважатимемо, що $f_1(x) \not\equiv 0$\\
Поділимо рівняння на $e^{\lambda_m x}$, отримавши:\\
$f_1(x)e^{(\lambda_1-\lambda_m) x} + f_2(x)e^{(\lambda_2-\lambda_m) x} + \dots + f_m(x) = 0$\\
Продиференціюємо таку кількість разів, щоб позбутись від $f_m(x)$.
Тут кожний доданок матиме наступний вираз:\\
$\displaystyle \frac{d^l}{dx^l} \left(f_p(x)e^{(\lambda_p - \lambda_m)x} \right) = \sum_{q=0}^l C_l^q f_p^{(q)}(x)(\lambda_p - \lambda_m)^{l-q}e^{(\lambda_p - \lambda_m)x} \overset{\textrm{позн}}{=} \tilde{f_p(x)}e^{(\lambda_p - \lambda_m)x}$\\
Зокрема для $k=1$:\\ $\displaystyle \tilde{f_1(x)} = \underbrace{f_1(x)(\lambda_1 - \lambda_m)^l}_{\neq 0} + \sum_{q=1}^l C_l^q f_1^{(q)}(x)(\lambda_1 - \lambda_m)^{l-q}e^{(\lambda_1 - \lambda_m)x} \neq 0$\\
Отримаємо, що:\\
$\tilde{f_1(x)}e^{(\lambda_1 - \lambda_m)x} + \tilde{f_2(x)}e^{(\lambda_2 - \lambda_m)x} + \dots + \tilde{f_{m-1}(x)}e^{(\lambda_{m-1} - \lambda_m)x} = 0$\\
Вийшла така ж сама тотожнсть по формі, що й з самого початку.\\
Якщо продовжити за MI, то прийдемо до тотожності:\\
$\overset{\vdots}{\tilde{f_1(x)}}e^{\lambda_1 x} = 0$\\
Але за умовою, $f_1(x) \neq 0$, що суперечить нашему припущенню 
\end{proof}

Ці 3 леми й завершують доведення теореми 
\end{proof}

\begin{example}
Розв'язати рівняння: $y^{(8)}+8y^{(6)}+16y^{4} = 0$
Запишемо характеристичний поліном:\\
$P(\lambda) = \lambda^8 + 8 \lambda^6 + 16 \lambda^4 = 0$\\
Звідси:\\
$\lambda_1 = 0$ - кратність 4, тому тут система: $\{1, x, x^2, x^3\}$\\
$\lambda_2 = 2i, \lambda_3 = -2i$ - кратності 2, тому тут система: $\{\cos 2x, \sin 2x, x \cos 2x, x \sin 2x \}$\\
Отже, $y = C_1 + C_2 x + C_3 x^2 + C_4 x^3 + C_5 \cos x + C_6 \sin x + C_7 x \cos x + C_8 x \sin x$
\end{example}

\subsection{Неоднорідне рівняння з постійними коефіцієнтами}
Спробуємо розв'язати рівняння
\begin{align*}
y^{(n)} + a_{n-1}y^{(n-1)}+\dots+a_1y'+a_0y = b(x)
\end{align*}
де $a_0, a_1,\dots,a_{n-1} \in \mathbb{R}, b \in C(I)$\\
В нашому випадку ми будемо розглядати $b(x) = e^{\sigma x} (\underset{= T_m(x)}{b_0 + b_1 x + \dots + b_m x^m} )$. Тут $\sigma \in \mathbb{R}$, а також $b_m \neq 0$.
І нехай характеристичний поліном $P(\lambda) = (\lambda - \lambda_1)^{r_1} \dots (\lambda - \lambda_k)^{r_k}$\\
\\
 Для правої частини існує 3 унікальних випадків\\
\textbf{I. Нерезонансний випадок}
\begin{theorem}
Якщо $P(\sigma) \neq 0$, то існує частковий розв'язок рівняння такого вигляду:\\
$y_{p.inh.} = e^{\sigma x} (\underset{= Q_m(x)}{q_0 + q_1 x + \dots + q_m x^m})$
\end{theorem}

\begin{proof}
За принципом суперпозиції, ми будемо мати, що:\\
$y_{p.inh} = y_0 + \dots + y_m$, де\\
$Ly_j = y^{(n)}_j + a_{n-1}y^{(n-1)}_j +\dots+a_1y'_j +a_0y_j = b_j x^j e^{\sigma x} , j = \overline{0,m}$\\
Розглянемо дію оператора $\displaystyle D_{\sigma} = \frac{d}{dx} - \sigma I$ на вираз $x^j e^{\sigma x}$. Отримаємо:\\
$\displaystyle D^j_{\sigma} \left( x^j e^{\sigma x} \right) = j!e^{\sigma x}$ (можна самому переконатись)\\
$\displaystyle D^{j+1}_{\sigma} \left( x^j e^{\sigma x} \right) = 0$\\
Останній оператор ми застосуємо до обох частин рівняння:\\
$\displaystyle D^{j+1}_{\sigma} (Ly) = b_j \displaystyle D^{j+1}_{\sigma} \left( x^j e^{\sigma x} \right) = 0$ - однорідне лінійне рівняння []\\
Підставимо $y = e^{\lambda x}$ \\
$\displaystyle D^{j+1}_{\sigma} (Le^{\lambda x}) = \displaystyle D^{j+1}_{\sigma} (P(\lambda)e^{\lambda x}) = P(\lambda) (\lambda - \sigma)^{j+1}e^{\lambda x} = 0$\\
Тоді характеристичний поліном для [] $P'(\lambda) = P(\lambda)(\lambda-\sigma)^{j+1} = 0$\\
Корені: $\sigma, \lambda_1, \dots, \lambda_k$, кратність $r_1, \dots, r_k$. За умовою, $P(\sigma) \neq 0$, а тому $\sigma \neq \lambda_l, l = \overline{1,k}$.\\
Фундаментальна система: до цього + $\{e^{\sigma x}, x e^{\sigma x}, \dots, x^j e^{\sigma x} \}$
\end{proof}
\newpage


\section{Диференціальні рівняння, що не потрапили}
\subsection{Рівняння $y' = f(ax+by+c)$}
Маємо таке рівняння:
\begin{align*}
y' = f(ax+by+c)
\end{align*}
Якщо $b = 0$, то тоді ми прийдемо до рівняння з відокремленими змінними\\
Робимо заміну
\begin{align*}
z(x) = ax + by + c
\end{align*}
Тоді $z' = a + by'$\\
$\Rightarrow \dfrac{z'-a}{b} = f(z)$\\
$\Rightarrow z' = bf(z) + a$ - рівняння з відокремленими змінними...

\begin{example}
 Розв'язати рівняння: $y' = (x+y+1)^2$\\
Заміна: $z = x+y+1 \Rightarrow z' = 1 + y'$\\
$z' - 1 = z^2 \Rightarrow z' = z^2 + 1 \Rightarrow \dfrac{dz}{z^2+1} = dx \Rightarrow \arctg z = x + C$\\
Проводимо зворотню заміну:\\
$\arctg(x+y+1) = x + C$
\end{example}

\subsection{Квазіоднорідні рівняння}
Маємо стандартне диф. рівняння
\begin{align*}
y' = f(x,y)
\end{align*}
І нехай додатково для функції $f(x,y)$ виконується така властивість
\begin{align*}
\exists \sigma \in \mathbb{R}: \forall t \neq 0: f(tx, t^\sigma y) = t^{\sigma - 1} f(x,y)
\end{align*}
Якщо $\sigma = 1$, то ми повертаємось до однорідних рівнянь\\
Робимо заміну
\begin{align*}
y = z \cdot x^\sigma
\end{align*}
Тоді $y' = z' x^\sigma + \sigma z x^{\sigma - 1}$\\
$\Rightarrow z' x^\sigma + \sigma z x^{\sigma - 1} =  f(x, zx^\sigma)$\\
Оскільки маємо квазіоднорідне рівняння, то $f(x,zx^\sigma) = x^{\sigma - 1} f(1,z)$\\
$\Rightarrow z'x + \sigma z = f(1,z) \Rightarrow z' = \dfrac{f(1,z) - \sigma z}{x}$\\
Це вже рівняння з відокремленими змінними

\begin{example}
 Розв'язати рівняння $y' = y^2 + \dfrac{1}{4x^2}$\\
Перевірка на квазіоднорідність:\\
$f(tx, t^\sigma y) = t^{2 \sigma} y^2 + \dfrac{1}{4t^2 x^2} \overset{?}{=} t^{\sigma -1} \left(y^2 + \dfrac{1}{4x^2} \right)$\\
$\begin{cases}
2\sigma = \sigma - 1 \\
-2 = \sigma - 1
\end{cases}
$\\
Отже, маємо, що $\sigma = -1$\\
Заміна: $y = zx^{-1} \Rightarrow y' = \dfrac{z'}{x} - \dfrac{z}{x^2}$\\
$\Rightarrow \dfrac{z'}{x} - \dfrac{z}{x^2} = \dfrac{z^2}{x^2} + \dfrac{1}{4x^2}
\Rightarrow z'x - z = z^2 + \dfrac{1}{4} \Rightarrow z' = \dfrac{z^2+z+\dfrac{1}{4}}{x}$\\
$\Rightarrow \dfrac{dz}{dx} = \dfrac{\left( z+\dfrac{1}{2} \right)^2}{x} \Rightarrow -\dfrac{1}{z + \dfrac{1}{2}} = \ln |x| + C$\\
Проводимо зворотню заміну:\\
$-\dfrac{2}{2xy+1} = \ln |x| + C$
\end{example}

\begin{remark}
 Квазіоднорідні рівняння можуть скоротити область визначення. Наприклад, якщо $\sigma = \dfrac{1}{2}$, то ми маємо розв'язки лише для $x > 0$\\
Тоді можна застосувати заміну $x = -p$, коли $x < 0$
\end{remark}

\subsection{Лінійне рівняння методом Ейлера}
Розглядується рівняння наступного вигляду:
\begin{align*}
y' + a(x)y = b(x)
\end{align*}
Домножимо обидві частини рівняння на $e^{\int a(x)\,dx}$, маємо:\\
$y' e^{\int a(x)\,dx} + a(x)y e^{\int a(x)\,dx} = b(x)e^{\int a(x)\,dx}$\\
$\left( y e^{\int a(x)\,dx} \right)' = b(x) e^{\int a(x)\,dx}$\\
$y e^{\int a(x)\,dx} = \displaystyle \int b(x) e^{\int a(x)\,dx} \,dx + C$\\
$y = e^{-\int a(x)\,dx} \left(\displaystyle \int b(x) e^{\int a(x)\,dx} \,dx + C \right)$

\begin{example}
 Розв'язати рівняння: $y' - \dfrac{2y}{x} = 2x^3$\\
Тут $a(x) = -\dfrac{2}{x} \Rightarrow \displaystyle \int a(x)\,dx = -2 \ln x = -\ln x^2$\\
А далі множимо обидві частини рівняння на $e^{\int a(x)\,dx} = e^{-\ln x^2} = \dfrac{1}{x^2}$\\
$\Rightarrow \dfrac{y'}{x^2} - \dfrac{2y}{x^3} = 2x \Rightarrow \left( \dfrac{y}{x^2} \right)' = 2x \Rightarrow \dfrac{y}{x^2} = x^2 + C \\ \Rightarrow y = x^4 + Cx^2$
\end{example}

\subsection{Рівняння Рікатті}
Розглянемо таке рівняння
\begin{align*}
y' = P(x)y^2 + Q(x)y + R(x)
\end{align*}
де $P,Q,R \in C(I)$

\begin{remark}
 При $P(x) \equiv 0$ маємо лінійне рівняння\\
При $R \equiv 0$ маємо рівняння Бернуллі, де $\lambda = 2$\\
При $P = Q = R = const$ маємо рівняння з відокремленими змінними
\end{remark}

Проведемо заміну
\begin{align*}
y = z + y_{part}
\end{align*}
де $z = z(x)$ - така функція, що зможе звести до рівняння Бернуллі, а $y_{part}$ - якийсь частковий розв'язок\\
$\Rightarrow y' = z' + y'_{part} = z' + P(x)y_{part}^2 + Q(x)y_{part} + R(x)$\\
Підставимо це в наше рівняння:\\
$z' + P(x)y_{part}^2 + Q(x)y_{part} + R(x) = P(x)(z + y_{part})^2 + Q(x)(z + y_{part}) + R(x)$\\
Якщо трохи поскоротити, отримаємо таке рівняння:\\
$z' = P(x)z^2 + (2P(x)y_{part} + Q(x))z$\\
А це вже - рівняння Бернуллі з $\lambda = 2$...

\begin{example}
 Розв'язати рівняння $y' - 2xy+y^2 = 5-x^2$\\
Або $y' = -y^2 + 2xy + (5-x^2)$\\
Спробуємо вгадати розв'язок у вигляді $y_{part} = kx + b \Rightarrow y'_{part} = k$\\
$\Rightarrow k = -(kx+b)^2 + 2x(kx+b) + (5-x^2)$\\
$\Rightarrow k = -k^2x^2 - 2kxb - b^2 + 2kx^2 + 2bx + 5 - x^2$\\
$\Rightarrow (k^2-2k)x^2 + (2kb-2b)x + (k+b^2) = -x^2 + 5$\\
$\Rightarrow \begin{cases}
k^2 - 2k = -1\\
2kb - 2b = 0\\
k+b^2=5
\end{cases} \Rightarrow \begin{cases} k = 1 \\ b = \pm 2 \end{cases}$\\
Візьмемо $y_{part} = x + 2$
\bigskip \\
Заміна: $y = z + x + 2 \Rightarrow y' = z' + 1$\\
$\Rightarrow z' + 1 = -(z+x+2)^2 + 2x(z+x+2) + 5 - x^2$\\
$\Rightarrow z' = -z^2 - 4z$\\
$\Rightarrow z' + 4z = -z^2$ - рівняння Бернуллі, $\lambda = 2$\\
...\\
$z = -\dfrac{4C'}{C' - e^{4x}}$\\
Зворотня заміна:\\
$y-x-2 = -\dfrac{4C'}{C' - e^{4x}}$
\end{example}

\subsection{Канонічний вигляд рівняння Рікатті}
Маємо таке рівняння
\begin{align*}
y' = y^2 + \tilde{Q}(x)
\end{align*}
Виявляється, що будь-яке рівняння Рікатті можна звести до канонічного вигляду\\
Для цього проведемо заміну:
\begin{align*}
y = \alpha(x) z(x)
\end{align*}
де $\alpha(x)$ - така функція, щоб коефіцієнт при $z^2$ був рівний $1$\\
$y' = \alpha'z + \alpha z'$\\
$\Rightarrow \alpha'z + \alpha z' = P(x) \alpha^2(x) z^2(x) + Q(x) \alpha(x) z(x) + R(x)$\\
Поділимо на $\alpha$ та виразимо $z'$\\
$z' = P \alpha z^2 + \left(Q - \dfrac{\alpha'}{\alpha} \right)z + \dfrac{R}{\alpha}$\\
$P \alpha = 1$ згідно з заміною\\
Візьмемо $\alpha(x) = \dfrac{1}{P(x)}$. Тоді наша перша заміна вже матиме вигляд:\\
$y = \dfrac{z(x)}{P(x)}$\\
Проведемо другу заміну
\begin{align*}
z = u(x) + \beta(x)
\end{align*}
де $\beta(x)$ - така функція, щоб коефіцієнт при $u$ був рівний $0$\\
$z' = u' + \beta'$\\
$\Rightarrow u' + \beta' = (u+\beta)^2 + \left(Q - \dfrac{\alpha'}{\alpha} \right)(u+\beta) + \dfrac{R}{\alpha}$\\
Виразимо $u'$\\
$u' = u^2 + u \left( 2\beta + Q - \dfrac{\alpha'}{\alpha} \right) + \beta^2 + \left(Q - \dfrac{\alpha'}{\alpha} \right) \beta + \dfrac{R}{\alpha} - \beta'$\\
$2\beta + Q - \dfrac{\alpha'}{\alpha} = 0$ згідно з заміною\\
$\beta^2 + \left( Q - \dfrac{\alpha'}{\alpha} \right)\beta + \dfrac{R}{\alpha} - \beta' = \tilde{Q}(x)$\\
І нарешті, візьмемо $\beta = \dfrac{1}{\alpha} \left( \dfrac{\alpha'}{\alpha} - Q \right)$, де $\alpha = \dfrac{1}{P(x)}$\\
$\Rightarrow u' = u^2 + \tilde{Q}(x)$ - рівняння Рікатті в канонічному вигляді

\begin{example}
 Звести до канонічного рівняння Рікатті $y' = \dfrac{y^2}{x^2} + \dfrac{2y}{x} + \dfrac{3}{x^2}$\\
$P(x) = \dfrac{1}{x^2}, Q(x) = \dfrac{2}{x}, R(x) = \dfrac{3}{x^2}$\\
Заміна 1: $y = \dfrac{z(x)}{P(x)} = x^2 z \Rightarrow y' = x^2 z' + 2xz$\\
$\Rightarrow x^2 z' + 2xz = x^2 z^2 + 2xz + \dfrac{3}{x^2} \Rightarrow x^2z' = x^2z^2 + \dfrac{3}{x^2}$\\
$\Rightarrow z' = z^2 + \dfrac{3}{x^4}$ - канонічне рівняння
\end{example}

\subsection{Спеціальні рівняння Рікатті}
Маємо таке рівняння
\begin{align*}
y' + Ay^2 = Bx^m
\end{align*}
де $A,B, m \in \mathbb{R}$

\begin{remark}
 При $m = 0$ маємо рівняння з відокремленними змінними\\
При $m = -2$ маємо квазіоднорідне рівняння, де $\sigma = -1$
\end{remark}

\begin{theorem} 
Спеціальне рівняння Рікатті є інтегрованим $\iff$ $\dfrac{m}{2m+4} \in \mathbb{Z}$\\
\textit{Факт доволі складний}
\end{theorem}

Розглянемо випадок, коли $\dfrac{m}{2m+4} \in \mathbb{Z}$\\
Зробимо заміну
\begin{align*}
y = \dfrac{z(x)}{x}
\end{align*}
$\Rightarrow y' = \dfrac{z'}{x} - \dfrac{z}{x^2}$\\
$\Rightarrow \dfrac{z'}{x} - \dfrac{z}{x^2} + A \dfrac{z^2}{x^2} = Bx^m$\\
$\Rightarrow z'x - z + Az^2 = Bx^{m+2}$\\
Зробимо другу заміну
\begin{align*}
x^{m+2} = t
\end{align*}
де $t$ - невідома змінна\\
$\Rightarrow \dfrac{dz}{dx} = \dfrac{dz}{dt} \dfrac{dt}{dx} = \dfrac{dz}{dt} (m+2)x^{m+1}$\\
$\Rightarrow x \dfrac{dz}{dx} = (m+2) \dfrac{dz}{dt} x^{m+2} = (m+2)t \dfrac{dz}{dt}$\\
$\Rightarrow (m+2)t \dfrac{dz}{dt} - z + Az^2 = Bt$\\
$t \dfrac{dz}{dt} - \dfrac{1}{m+2}z + \dfrac{A}{m+2}z^2 = \dfrac{Bt}{m+2}$\\
Отримали рівняння Рікатті вигляду\\
$t \dfrac{dz}{dt} + \alpha z + \beta z^2 = \gamma t$\\
В залежності від ситуації виконаємо одну з двох замін
\begin{align*}
z(t) = \dfrac{t}{a+u(t)}, \alpha < -\dfrac{1}{2}
\end{align*}
де $a = \dfrac{1+\alpha}{2}$\\
Або
\begin{align*}
z(t) = -\dfrac{\alpha}{\beta} + \dfrac{t}{u(t)}, \alpha > -\dfrac{1}{2}
\end{align*}
Такі заміни робимо стільки разів, скільки потрібно, поки не отримуємо ще одне рівняння Рікатті\\
$u't - \dfrac{1}{2}u + Du^2 = Ht$\\
Зробимо останню заміну
\begin{align*}
u = v(t) \sqrt{t}
\end{align*}
$\Rightarrow u' = v' \sqrt{t} + \dfrac{v \sqrt{t}}{2}$\\
$\Rightarrow u' t \sqrt{t} + Dv^2t = Ht$\\
$v' \sqrt{t} = H - Dv^2$ - рівняння з відокремленими змінними
\bigskip \\

\subsection{Диференціальні рівняння в симетричній формі}
Маємо рівняння Пфаффа
\begin{align*}
M(x,y)\,dx = N(x,y)\,dy = 0
\end{align*}
де $M,N: D \to \mathbb{R}, D \subset \mathbb{R}^2, M,N \in C(D)$, а також\\
$|M(x,y)|+|N(x,y)| \not\equiv 0$

\begin{definition}
Крива $x=x(t), y=y(t)$, $t \in I$ є \textbf{розв'язком} заданого рівняння, якщо
\begin{align*}
x(t),y(t) \in C'(I) \\
\forall t \in I: ((x(t),y(t)) \in D \\
M(x(t)),y(t))x'(t) + N(x(t),y(t))y'(t) \equiv 0
\end{align*}
\end{definition}

\begin{definition}
Вираз $F(x,y,c) = 0$ задає \textbf{загальний розв'язок} заданого рівняння, якщо будь-який розв'язок кривої може бути представлений у такому вигляді
\end{definition}

\subsubsection{Рівняння в повних диференціалах}
\begin{definition}
Рівняння Пфаффа називається \textbf{рівнянням в повних \\ диференціалах}, якщо
\begin{align*}
\exists u(x,y) \in C'(D): \dfrac{\partial u}{\partial x} = M(x,y), \dfrac{\partial u}{\partial y} = N(x,y)
\end{align*}
\end{definition}

Тоді рівняння прийме такий вигляд
\begin{align*}
\dfrac{\partial u}{\partial x}\,dx + \dfrac{\partial u}{\partial y}\,dy = 0 \iff du = 0 \iff u(x,y)=c
\end{align*}

\begin{theorem}[Критерій рівняння в повних диференціалах]
$M(x,y)\,dx + N(x,y)\,dy = 0$ - в повних диференціалах $\iff \dfrac{\partial M}{\partial y} = \dfrac{\partial N}{\partial x}$
\textit{Доведення див. в мат анализі 3 семестр}
\end{theorem}

\begin{example}
 Розв'язати рівняння $(x^2+y)\,dx + (x+y^2)\,dy = 0$\\
Оскільки $\dfrac{\partial M}{\partial y} = \dfrac{\partial N}{\partial x} = 1$, то таке рівняння - в повних диференціалах\\
Отже, $\exists u(x,y): \dfrac{\partial u}{\partial x} = M(x,y), \dfrac{\partial u}{\partial y} = N(x,y)$\\
$\Rightarrow u(x,y) = \displaystyle \int M(x,y)\,dx + \varphi(y) = \int (x^2+y)\,dx + \varphi(y) = \\ = \dfrac{x^3}{3} + yx + \varphi(y)$\\
$\Rightarrow \dfrac{\partial u}{\partial y} = N(x,y) = \dfrac{\partial}{\partial y} \left( \displaystyle \int M(x,y)\,dx + \varphi(y) \right)$\\
$\Rightarrow x + y^2 = x + \varphi'(y) \Rightarrow \varphi'(y) = y^2 \Rightarrow \varphi(y) = \dfrac{y^3}{3}$\\
Остаточно $u(x,y) = \dfrac{x^3}{3} + xy + \dfrac{y^3}{3} + C$
\end{example}

\subsubsection{Інтегрувальний множник}
Тепер розглянемо рівняння Пфаффа, але вже не в повних диференціалах\\
Проте завжди можна звести до повних диференціалах шляхом домноження на деяку неперервну функцію $\mu(x,y)$

\begin{example}
 $y\,dx - x\,dy = 0$, не є рівняннях в повних диференціалах, тому що\\
$\dfrac{\partial M}{\partial y} = 1, \dfrac{\partial N}{\partial x} = -1$\\
Проте якщо рівняння домножити на $\mu(x,y) = \dfrac{1}{y^2}$, то тепер\\
$\dfrac{1}{y}\,dx - \dfrac{x}{y^2}\,dy = 0$\\
$\dfrac{\partial M}{\partial y} = \dfrac{\partial N}{\partial x} = -\dfrac{1}{y^2}$
\end{example}

\begin{definition}
Функція $\mu(x,y)$ називається \textbf{інтегрувальним множником}, якщо при множенні на рівняння Пфаффа ми отримуємо рівняння в повних диференціалах
\end{definition}

З'ясуємо, як це знайти:\\
$\dfrac{\partial (\mu M)}{\partial y} = \dfrac{\partial(\mu N)}{\partial x} \iff \dfrac{\partial \mu}{\partial y} M + \mu \dfrac{\partial M}{\partial y} = \dfrac{\partial \mu}{\partial x} N + \mu \dfrac{\partial N}{\partial x}$\\
Рівняння справа й дасть відповідь на те, який множник нам треба
\bigskip \\
Часткові випадки:\\
1. $\mu = \mu(x)$\\
Тоді $\dfrac{\partial \mu}{\partial y} M + \mu \dfrac{\partial M}{\partial y} = \dfrac{\partial \mu}{\partial x} N + \mu \dfrac{\partial N}{\partial x} \iff \mu \dfrac{\partial M}{\partial y} = \dfrac{\partial \mu}{\partial x} N + \mu \dfrac{\partial N}{\partial x}$\\
$\iff \dfrac{\partial \mu}{\partial x} \dfrac{1}{\mu} = \dfrac{\dfrac{\partial M}{\partial y} - \dfrac{\partial N}{\partial x}}{N}$
Оскільки ліва функція лише залежить від $x$, то права частина рівняння водночас теж має лише залежати від $x$. Отже, $\mu(x)$ буде знайдено, інтегруючи це рівняння
\bigskip \\
2. $\mu = \mu(y)$\\
Аналогічним чином, не буду розписувати

\begin{example}
 Розв'язати рівняння $(x^2+y^2+x)\,dx + y\,dy = 0$\\
Маємо, що $\dfrac{\partial M}{\partial x} = 2y, \dfrac{\partial N}{\partial x} = 0$\\
Тоді $\dfrac{\partial M}{\partial y} - \dfrac{\partial N}{\partial x} = 2y$\\
$\Rightarrow \dfrac{\dfrac{\partial M}{\partial y} - \dfrac{\partial N}{\partial x}}{N} = 2 = \mu(x) \Rightarrow \exists \mu = \mu(x)$\\
$\Rightarrow \dfrac{\partial \mu}{\partial x} \dfrac{1}{\mu (x)} = 2 \Rightarrow \mu = e^{2x}$\\
Домножимо рівняння на інтегрувальний множник:\\
$e^{2x}(x^2+y^2+x)\,dx + e^{2x}y\,dy = 0$\\
Тепер це - рівняння в повних диференціалах...
\end{example}

Рівняння Пфаффа також має множники $\mu = \mu(\omega(x,y))$ (наприклад, \\ $\mu(x+y), \mu(xy), \mu(x^2+y^2),\dots$), але то таке

\end{document}
